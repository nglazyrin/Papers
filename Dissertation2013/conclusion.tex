\chapter*{Заключение}						% Заголовок
\addcontentsline{toc}{chapter}{Заключение}	% Добавляем его в оглавление
\label{chaptC}

Основные результаты работы заключаются в следующем.
\begin{enumerate}
  \item На основе анализа свойств музыкальных звукозаписей был разработан метод
  для более точного выделения в звуке компонент, соответствующих музыкальным
  инструментам.
  \item Исследование показало, что глубокие нейронные сети в применении к
  получению музыкальных признаков могут показывать хорошие результаты, сравнимые
  с результатами традиционных подходов к получению признаков.
  \item Численные исследования показали, что реализованные в рамках работы
  подходы позволяют добиться качества распознавания аккордов, сравнимого с
  лучшими из известных алгоритмов, при достаточно высокой скорости обработки.
  \item Для выполнения поставленных задач был создан программный комплекс на
  языках Java и Python, свободно доступный через интернет.
\end{enumerate}

В нынешних условиях любые методы обработки информации должны быть нацелены на
работу с очень большими её объёмами. Вероятно, именно чрезмерное количество
времени, необходимое для анализа миллионов композиций, препятствует широкому
применению методов распознавания аккордов. С другой стороны, малый объем
размеченных и доступных для обучения данных затрудняет широкое применение
методов, основанных на алгоритмах машинного обучения. Именно поэтому, как
представляется автору, разработка достаточно быстрого метода распознавания
аккордов, не использующего машинное обучение, является важным шагом на пути к
обработке больших музыкальных коллекций.

\clearpage