\chapter*{Введение}							% Заголовок
\addcontentsline{toc}{chapter}{Введение}	% Добавляем его в оглавление

% Связь музыки и математики была обнаружена ещё в Античные времена. Пифагор
% заметил, что частоты звуков, составляющих наиболее значимые гармоничные
% интервалы или консонансы, относятся друг к другу как целые числа: 2:1 (октава),
% 3:2 (квинта), 4:3 (кварта) (см. \cite{Christensen2002}, с. 274). Понятия
% окружности и прямой линии позволили представить взаимоотношения мажорных и
% минорных тональностей и консонантных интервалов на плоскости (см.
% \cite{Christensen2002}, с. 283). Модульная арифметика и сложение по модулю 12
% позволили математически выразить эквивалентность нот с одним и тем же названием
% в разных октавах, обращение и транспозицию интервалов. Это отношение
% эквивалентности делит все музыкальные звуки на классы эквивалентности, которые
% называют \emph{высотными классами}. В свою очередь, операции транспозиции и
% обращения задают группу преобразований на множестве высотных классов
% (\cite{Christensen2002}, с. 290). Используя свойства этого множества и его
% булеана, также можно выразить музыкальные взаимоотношения в музыке различных
% эпох и культур. Ещё более абстрактная модель \emph{обобщённой интервальной
% системы} задается тройкой: набор музыкальных объектов, группа обобщённых
% интервалов и функция, сопоставляющая каждой паре объектов обобщённый интервал.
% Такая модель позволяет представить всевозможные музыкальные пространства (см.
% \cite{Christensen2002}, с. 295-296).

С давних времён в области музыки возникают задачи воспроизведения (в том числе
повторного), записи, хранения, классификации, поиска. В настоящее время все они
решаются в том числе с помощью компьютеров. Тематика данной работы относится к
областям классификации и поиска музыки.

На текущий момент компьютер является основным средством для хранения и обработки
музыки и любой информации о музыке, будь то ноты, биография композитора, год
выпуска записи или график концертов группы. Сама по себе музыка, содержащаяся в
цифровых звукозаписях, более ценна для человека, поскольку никакая
информация о ней не может заменить собой её прослушивание. Вместе с тем, именно
эта дополнительная информация даёт возможность ориентироваться в музыкальных
коллекциях, находить новую музыку, организовывать существующие записи. В силу
большей ценности музыки, зачастую звукозаписи не сопровождаются дополнительной
информацией. Необходимость получения разнообразной информации о данной цифровой
звукозаписи порождает множество задач, связанных с обработкой звука:
идентификация композиции, нахождение разных версий одной композиции, определение
заданной композиции в потоке звука с радио, поиск похожих композиций,
определение мелодии композиции для последующего воспроизведения на музыкальном
инструменте и другие. Эта диссертация посвящена задаче определения
последовательности аккордов в звуке.

Одним из способов решения названных задач является автоматизированное при помощи
компьютера извлечение мелодии и/или аккордов из цифровой записи звука с
последующим использованием извлеченной информации для индексации, поиска,
сравнения. Данная диссертация посвящена задаче определения последовательности
аккордов в звуке, представленном в цифровом виде.

Аккорды -- это основная информация, необходимая гитаристу для того, чтобы
сыграть композицию. Многочисленные гитаристы-любители, не имеющие достаточно
опыта или усидчивости, чтобы самостоятельно определить звучащие аккорды, смогут
получить инструмент, решающий эту задачу за них. Информация о последовательности
аккордов может быть использована также для индексации композиций и последующего
поиска по запросу. Среди возможных сценариев такого поиска можно отметить
следующие:
\begin{itemize}
  \item поиск заимствований или разных версий одной и той же композиции;
  \item поиск композиций, которые могут гармонично сочетаться друг с другом.
\end{itemize}

Последовательность аккордов может быть представлена в текстовом виде. С одной
стороны, представление звука в виде последовательности символов позволяет
перенести задачи индексации и поиска музыки в хорошо проработанную область
индексации и поиска текстов. С другой стороны, такое представление композиции
позволяет человеку вводить поисковые запросы без использования звуковых файлов.

\medskip

Первые попытки обработки музыкальной информации в символьном виде были сделаны в
1950-х годах с появлением первых компьютеров. Они были связаны с автоматическим
определением закономерностей в музыке с целью использования их для создания
новых мелодий (см. \cite{Schueler2005}). Тогда же было предложено использовать
компьютер для распознавания и печати нотных записей, анализа схожести различных
композиций и поиска по образцу. В 1960-х годах появляются первые работы
(например, \cite{Freedman1967}), связанные с анализом звукозаписей,
представленных в цифровом виде. Их целью было, прежде всего, понимание того, из
чего состоят воспроизводимые музыкальными инструментами звуки и как они
воспринимаются человеком.

В 1975 году в \cite{Moorer1975} было положено начало новому применению
компьютера к анализу цифровых музыкальных звукозаписей: распознаванию в них
отдельных нот. Этот процесс объединяют с компьютерным распознаванием нотных
записей под общим названием \emph{транскрибирование}. Здесь впервые теория
музыки используется для анализа композиции не в виде нотной записи, а в том
виде, в котором её воспринимает обычный слушатель -- в виде звукозаписи.
Несмотря на раннюю постановку и большое количество приложенных усилий, задача
транскрибирования музыкальной звукозаписи не решена до сих пор.

В 1982 году компаниями Sony и Philips было запущено массовое производство
компакт-дисков, на которых музыка была записана в цифровом формате. Со временем
доступных в цифровом виде произведений стало на порядки больше, чем доступных
нотных записей. Закономерно возрос интерес к автоматическому транскрибированию
музыки. В \cite{Moorer1975} рассматривались только звукозаписи, содержащие не
более двух одновременно звучащих музыкальных инструментов. В 1996 году в
\cite{Martin1996} был представлен один из первых методов, подходивших для любой
полифонической цифровой аудиозаписи.

Задача определения последовательности аккордов при этом не отделялась от задачи
транскрибирования. Как отмечает Т. Фуджишима в \cite{Fujishima1999},
в 1980-1990-х годах (например, в работе \cite{Aono1998}) проблема распознавания
аккордов в музыке решалась путём распознавания отдельных нот и их объединения в
аккорды. Он же впервые предложил метод распознавания аккордов без
предварительного транскрибирования звукозаписи. В \cite{Aono1998} метод
распознавания аккордов являлся частью системы для автоматического
аккомпанемента выступлению живого человека.

В 2000-х годах определение аккордов окончательно выделяется в отдельную задачу.
Начиная с 2008 года в рамках ежегодной кампании по оценке методов музыкального
информационного поиска MIREX \cite{MirexHome} проводятся соревнования среди
алгоритмов распознавания аккордов в звуке. За это время был достигнут
существенный прогресс в качестве распознавания. В 2012 году на это соревнование
были выставлены более 10 алгоритмов.

В 2010-х годах появляются широко доступные программные продукты, включающие в
себя такие алгоритмы. Популярный пакет для профессионального создания музыки
\emph{Ableton Live 9} \cite{AbletonLive} позволяет преобразовать любую
звукозапись, в том числе полифоническую, в нотное представление в редакторе. Эта
возможность может быть использована в первую очередь как альтернативный способ
ввода нотных данных для последующего редактирования.

Приложения для смартфонов \emph{AnySong Chord Recognition}
\cite{AnySongChordRecognition} и \emph{Chord Detector} \cite{ChordDetector} позволяют
определить аккорды в звуковом файле и показывают соответствующие гитарные
табулатуры, позволяя играть на гитаре композицию одновременно с её
воспроизведением. Соответственно, эти приложения нацелены в первую очередь на
использование с гитарной музыкой.

Интернет-сервис Chordify \cite{Chordify} позволяет определить последовательность
аккордов в произвольном видео с \url{http://youtube.com}, аудио с
\url{http://soundcloud.com} или в загруженной пользователем звукозаписи, после
чего воспроизвести звук или видео с одновременной индикацией звучащего аккорда.
Наряду с недостаточным качеством распознавания, недостатком этого продукта
является отсутствие возможности поиска по заданной последовательности аккордов.
На сегодняшний день автору не известны какие-либо продукты, предназначенные для
обработки коллекции разнообразных музыкальных звукозаписей с целью поиска
похожих или гармонично сочетающихся друг с другом композиций.

Музыкальные звуки имеют длительность, которая, как правило, существенно меньше
длительности всей композиции. Аккорд как совокупность звуков также имеет
определенную относительно небольшую длительность. Поэтому естественно
анализировать звукозапись, разделяя её на короткие фрагменты соразмерной длины.
На каждом фрагменте определяется набор признаков, по которому определяется
соответствующий аккорд. Итоговое качество распознавания зависит как от выбора
признаков, так и от алгоритма, сопоставляющего набору признаков аккорд.

Признаки позволяют представить в компактном по сравнению со звукозаписью виде
основную информацию о звуке на данном фрагменте. В отличие от
амплитудно-частотного представления, признаки не содержат дубирующейся
информации. Было предложено большое количество разнообразных алгоритмов
получения звуковых признаков, использующих особенности звучания музыкальных
инструментов, особенности человеческого восприятия и возможные помехи на
звукозаписях.

В 2010-х годах становятся чрезвычайно популярными так называемые методы
обучения представлениям. Они, фактически, являются алгоритмами со множеством
автоматически подбираемых параметров, позволяющими получить признаки, наилучшим
образом отражающие необходимую для дальнейшего использования информацию. За
исключением Хамфри \cite{Humphrey2012} никто не применял методы глубокого
обучения к распознаванию аккордов.

Наиболее простым способом определения аккорда по набору признаков является метод
ближайшего соседа: вычисление расстояний от заданного набора до <<идеальных>>,
шаблонных наборов признаков для каждого аккорда. При этом можно рассматривать
разные метрики в пространстве признаков.

Вероятностные модели позволяют найти в некотором смысле наилучшую из заданного
класса метрик. Большинство алгоритмов, представленных в рамках соревнований
MIREX Audio Chord Estimation, используют скрытую марковскую модель или
байесовскую сеть и моделируют последовательность векторов признаков как
марковский процесс. При этом наблюдениями модели являются признаки на каждом
фрагменте, а скрытыми состояниями -- соответствующие аккорды. Параметры моделей
настраиваются в процессе обучения на размеченных данных. Несмотря на достаточно
высокое качество распознавания аккордов, такого рода модели имеют свои
недостатки. Среди них Де Хаас в \cite{DeHaas2012} выделяет следующие:
\begin{itemize}
  \item Потребность в большом количестве данных для обучения. Подготовка таких
  данных весьма трудоёмка, а сами данные могут сильно разниться для разных
  стилей музыки, эпох, композиторов.
  
  \item Опасность переобучения. Модели с большим количеством параметров
  наилучшим образом подстраиваются под доступный набор обучающих данных, но
  непонятно, насколько хорошо они будут подходить для работы с данными не из
  обучающей выборки.
  
  \item Многомерность данных. Она приводит к экспоненциальному увеличению объема
  данных и времени их обработки, а также к росту необходимого объёма обучающей
  выборки.
  
  \item Недостаточное использование времени. Марковское свойство предполагает
  зависимость только от предыдущего фрагмента. Но музыкальная композиция
  зачастую имеет определённую, достаточно протяжённую по времени, структуру,
  которая не может быть отражена в модели.
  
  \item Существуют другие условия, которые также не могут быть выражены в рамках
  обучаемой модели. Например, это культурный или географический контекст или
  сложившиеся практики и правила создания музыки.
  
  \item Сложность интерпретации модели, оперирующей в большей степени
  искусственными, математическими, нежели музыкальными конструкциями.
\end{itemize}

Ещё одним недостатком является то, что упомянутые вероятностные методы хорошо
приспособлены для моделирования смены состояния (звучащего аккорда) и хуже --
для моделирования продолжительности нахождения в одном состоянии.

Перечисленные проблемы не могут быть разрешены в рамках модели, строящейся
исключительно путём обучения на реальных данных. Де Хаас предложил другую
модель, которая строится на основе правил западной тональной гармонии без
использования алгоритмов машинного обучения, а следовательно, менее подверженную
описанным выше недостаткам. Она допускает простую интерпретацию и может быть
использована для гармонического анализа композиции. Эта модель позволяет
корректировать последовательность, полученную после вычисления евклидовых
расстояний между векторами признаков и шаблонами аккордов. К сожалению, при
попытке применения модели ко всей последовательности расстояний требуется
перебор слишком большого количества вариантов. Поэтому требуется разделять
последовательность на короткие участки. Другим недостатком этой модели является
необходимость привязки к фрагментам, для которых считается, что аккорд
изначально был определён верно. Ошибка на таком фрагменте влечёт за собой ошибки
на соседних фрагментах.

Таким образом, разработка метода для распознавания последовательности аккордов,
не требующего большого объема данных для обучения, и не предполагающего
использования сложной многопараметрической самообучающейся модели, но при этом
сопоставимого по качеству результатов с уже существующими методами, является
вполне естественной и актуальной. Именно разработка такого метода стала целью
для автора данной работы.

С учётом описанных выше недостатков существующих подходов, для достижения
поставленной цели перед автором данной работы были поставлены следующие
задачи:

\begin{enumerate}
  \item разработать метод для более точного выделения в звуке компонент,
  соответствующих музыкальным инструментам, с целью улучшения существующих
  алгоритмов вычисления признаков по фрагменту звукозаписи;
  \item исследовать применимость некоторых универсальных методов обучения
  представлениям к получению музыкальных признаков;
  \item улучшить алгоритм определения аккорда по вектору признаков,
  использующий сопоставление с шаблонами аккордов;
  \item реализовать описанные алгоритмы в виде комплекса программ, позволяющего
  распознавать последовательность аккордов в поданном на вход звуковом файле;
  \item сравнить качество распознавания аккордов с аналогами, приняв участие в
  соревновании MIREX Audio Chord Estimation.
\end{enumerate}

В рамках данной работы все поставленные задачи были решены.

При решении поставленных задач в работе использованы методы математического
моделирования, спектральный анализ (для получения и обработки спектрограммы),
алгоритмы машинного обучения (для получения признаков), методы
объектно-ориентированного программирования и многопоточного программирования
(для реализации описанных методов и ускорения вычислений).

В диссертации получены следующие основные результаты, которые выносятся на
защиту.

\begin{enumerate}
  \item Новый метод распознавания последовательности аккордов в звукозаписи, не
  использующий алгоритмов машинного обучения.
  \item Новый метод представления звукозаписи в виде последовательности векторов
  признаков с применением многослойных очищающих автоассоциаторов.
  \item Сравнительный анализ результатов работы предлагаемых методов на
  коллекции из 319 звукозаписей, подтверждающий их эффективность.
  \item Реализующий предложенные методы комплекс программ на языках Java и
  Python, созданный в рамках данной работы.
\end{enumerate}

Разработанный метод распознавания последовательности аккордов может применяться
для анализа звукозаписей с целью их самостоятельного воспроизведения, с целью
поиска схожих музыкальных композиций. Метод не подвержен опасности переобучения
под конкретную музыкальную коллекцию.

Основные результаты диссертационной работы докладывались на всероссийской
научной конференции "Анализ Изображений, Сетей и Текстов" (Екатеринбург, 2012),
на всероссийской научной конференции "Анализ Изображений, Сетей и Текстов"
(Екатеринбург, 2013), на 9-й международной конференции по вычислениям в области
звука и музыки (Копенгаген, 2012), на 13-й конференции международного
сообщества по музыкальному информационному поиску (Порто, 2012).

Результаты изложены в 4 печатных изданиях, 1 из которых изданы в журналах,
рекомендованных ВАК, 3 -- в тезисах докладов всероссийских и международных
конференций. Алгоритм был выставлен на соревнование среди алгоритмов
распознавания аккордов MIREX Audio Chord Estimation 2012 \cite{ACEMrx},
\cite{ACEMcg}, проводимое международной лабораторией оценки систем музыкального
информационного поиска (International Music Information Retrieval Systems
Evaluation Laboratory) университета Иллинойса, США.

\textit{Журналы из перечня ведущих периодических изданий:}

Н. Ю. Глазырин: "О задаче распознавания аккордов в цифровых звукозаписях",
Известия Иркутского государственного университета, серия "Математика", 2013, Т.
6, № 2. с. 2-17.

\textit{Тезисы международных конференций:}

Nikolay Glazyrin, Alexander Klepinin: «Chord Recognition using Prewitt Filter
and Self-Similarity», Proceedings of the 9th Sound and Music Computing
Conference, Copenhagen, Denmark, 11-14 July, 2012, pp. 480-485.

\textit{Тезисы всероссийских конференций:}

Николай Глазырин, Александр Клепинин: «Выделение гармонической информации из
музыкальных аудиозаписей». Доклады всероссийской научной конференции "Анализ
Изображений, Сетей и Текстов" (АИСТ 2012), Москва, Национальный Открытый
Университет "Интуит", с. 159-168.

Николай Глазырин: «Применение автоассоциаторов к распознаванию
последовательностей аккордов в цифровых звукозаписях», Доклады всероссийской
научной конференции "Анализ Изображений, Сетей и Текстов" (АИСТ 2013), Москва,
Национальный Открытый Университет "Интуит", с. 199-203.

Все исследования, результаты которых изложены в данной работе, получены лично
соискателем в процессе научных исследований. Из совместных публикаций в
диссертацию включен лишь тот материал, который непосредственно принадлежит
соискателю.

\medskip

Диссертация состоит из~введения, четырех глав, заключения и~двух приложений.
Полный объем диссертации составляет ХХХ~страница с~ХХ~рисунками и~ХХ~таблицами.
Список литературы содержит ХХХ~наименований.

В главе \ref{chaptT} представлены необходимые для дальнейшего изложения сведения
из теории музыки. Делается формальная постановка и теоретические основы задачи
распознавания аккордов в музыке.

Глава \ref{chaptL} посвящена подробному обзору литературы по рассматриваемой
теме. 

В главе \ref{chapt1} описываются улучшения для алгоритмов вычисления векторов
признаков и получения аккорда по вектору признаков. Вычисление спектрограммы с
повышенным разрешением по времени и частоте с последующим применением
скользящего фильтра и прореживанием позволяет лучше сохранить компонеты спектра
звука, соответствующие звучанию музыкальных инструментов с определённой высотой
звучания. Это помогает получить векторы признаков, в большей степени сохраняющие
необходимую для определения аккорда информацию. Коррекция вектора признаков с
использованием наиболее схожих с ним других векторов, а также некоторые
эвристические правила для коррекции последовательности распознанных аккордов
дают возможность исправить некоторые ошибки определения звучащего аккорда.
Описанные улучшения позволили вплотную приблизить качество распознавания
аккордов к результатам алгоритмов, использующих обучаемые вероятностные модели.

В главе \ref{chapt2} описывается способ вычисления векторов признаков с
использованием различных вариантов многослойных автоассоциаторов.
Рассматриваются также рекуррентные многослойные автоассоциаторы, позволяющие
моделировать зависимость вектора признаков на текущем фрагменте от вектора
признаков на предыдущем фрагменте звукозаписи. Автору не удалось добиться
повышения качества распознавания аккордов с использованием признаков,
полученных с при помощи автоассоциаторов, в сравнении с признаками, алгоритмы
вычисления которых придуманы и настроены человеком.

В главе \ref{chapt3} описываются и анализируются результаты экспериментов.
Исследуется влияние параметров описанных алгоритмов на результат, а также
количественный вклад каждого из реализованных методов в повышение качества
распознавания аккордов.


% (Зачем это надо. MIR и краткая историческая справка с указанием имён,
% предшественники. Разные условия при разных предназначениях. Почему машинное
% обучение -- это плохо. Формулировка основной задачи. Краткое содержание работы.
% Основные результаты.)
% Распознавание аккордов больше всего полагается на теорию музыки. Определение
% тональности не помогает, т.к. много ошибок - в пределах тональности, соседи на
% квинтовом круге (см. 11 - All You Need Is Love). MIREX.
% Измерение. (12 Gool Old-Fashioned Lover Boy - пример, когда аккорд длится 1
% долю, 10 - Things We Said Today - пропускаем короткие аккорды).
% 
% Обзор, введение в тему, обозначение места данной работы в мировых исследованиях и т.п.





% \textbf{Целью} данной работы является \ldots
% 
% Для~достижения поставленной цели необходимо было решить следующие задачи:
% \begin{enumerate}
%   \item Исследовать, разработать, вычислить и т.д. и т.п.
%   \item Исследовать, разработать, вычислить и т.д. и т.п.
%   \item Исследовать, разработать, вычислить и т.д. и т.п.
%   \item Исследовать, разработать, вычислить и т.д. и т.п.
% \end{enumerate}
% 
% \textbf{Основные положения, выносимые на~защиту:}
% \begin{enumerate}
%   \item Первое положение
%   \item Второе положение
%   \item Третье положение
%   \item Четвертое положение
% \end{enumerate}
% 
% \textbf{Научная новизна:}
% \begin{enumerate}
%   \item Впервые \ldots
%   \item Впервые \ldots
%   \item Было выполнено оригинальное исследование \ldots
% \end{enumerate}
% 
% \textbf{Научная и практическая значимость} \ldots
% 
% \textbf{Степень достоверности} полученных результатов обеспечивается \ldots
% Результаты находятся в соответствии с результатами, полученными другими авторами.
% 
% \textbf{Апробация работы.}
% Основные результаты работы докладывались~на:
% перечисление основных конференций, симпозиумов и т.п.
% 
% \textbf{Личный вклад.} Автор принимал активное участие \ldots
% 
% \textbf{Публикации.} Основные результаты по теме диссертации изложены в ХХ
% печатных изданиях~\cite{bib1,bib2,bib3,bib4,bib5}, Х из которых изданы в
% журналах, рекомендованных ВАК~\cite{bib1,bib2,bib3}, ХХ --- в тезисах
% докладов~\cite{bib4,bib5}.
% 
% \textbf{Объем и структура работы.} Диссертация состоит из~введения, четырех
% глав, заключения и~двух приложений. Полный объем диссертации составляет
% ХХХ~страница с~ХХ~рисунками и~ХХ~таблицами. Список литературы содержит
% ХХХ~наименований.

\clearpage