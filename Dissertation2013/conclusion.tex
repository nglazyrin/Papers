\chapter*{Заключение}						% Заголовок
\addcontentsline{toc}{chapter}{Заключение}	% Добавляем его в оглавление
\label{chaptC}

Основные результаты работы заключаются в следующем.
\begin{enumerate}
  \item На основе анализа свойств музыкальных звукозаписей был разработан метод
  для более точного выделения в спектре звука компонент, соответствующих
  музыкальным инструментам.
  \item Исследование показало, что глубокие нейронные сети в применении к
  получению музыкальных признаков для распознавания аккордов могут показывать
  хорошие результаты, сравнимые с результатами традиционных подходов к
  получению признаков.
  \item Численные исследования показали, что реализованные в рамках работы
  подходы позволяют добиться качества распознавания аккордов, сравнимого с
  лучшими из известных алгоритмов, при достаточно высокой скорости обработки.
  \item Для выполнения поставленных задач был создан программный комплекс на
  языках Java и Python, свободно доступный через интернет.
\end{enumerate}

Разработанный в рамках работы метод для более точного выделения в спектре звука
компонент, соответствующих музыкальным инструментам, может быть использован как
часть других алгоритмов музыкального информационного поиска, работающих с
содержимым звукозаписей. Так, в задаче определения мелодии более точная
спектрограмма позволит аккуратнее выделить воспроизведённые ноты. Алгоритмы для
определения тональности и нахождения повторений в музыке также выиграют от
повышения качества используемых спектральных данных.

Хроматические признаки, полученные из спектра с использованием предварительно
обученных многослойных нейронных сетей, позволяют добиться по крайней мере
такого же качества распознавания аккордов, как и в случае использования
традиционных хроматических признаков. Однако, в отличие от традиционных подходов
к получению признаков, для использования нейронной сети необходимо некоторое
количество размеченных данных, а также существенное время на её обучение. Это
делает данный подход значительно менее удобным для применения. Безусловно,
глубокое обучение является весьма перспективной областью для дальнейших
исследований, но его применение к звуковым данным пока не приводит к заметному
скачку в качестве работы алгоритмов.

В нынешних условиях любые методы обработки информации должны быть нацелены на
работу с очень большими её объёмами. Вероятно, именно чрезмерное количество
времени, необходимое для анализа миллионов композиций, препятствует широкому
применению методов распознавания аккордов. С другой стороны, малый объём
размеченных и доступных для обучения данных затрудняет широкое применение
методов, основанных на алгоритмах машинного обучения. Именно поэтому, как
представляется автору, разработка достаточно быстрого метода распознавания
аккордов, не использующего машинное обучение, является важным шагом на пути к
обработке больших музыкальных коллекций. Реализованный в рамках работы алгоритм,
использующий многопоточную обработку, является достаточно быстрым по сравнению с
аналогами. Но для более комфортной работы пользователей с программой было бы
желательно снизить время обработки одной композиции до нескольких секунд.

Кроме того, использованные в работе подходы могут быть применены в работе
компаний, имеющих дело с многочисленными цифровыми звукозаписями: музыкальные
магазины, музыкальные потоковые сервисы, площадки для размещения собственных
музыкальных записей. Как правило, такие компании обладают достаточными
вычислительными мощностями для обработки большого набора звуковых файлов, а от
алгоритмов не требуется работы в реальном времени. Анализ содержимого 
звукозаписей может улучшить качество поиска и рекомендации музыки в сервисах
этих компаний, предоставить новые критерии для поиска, дать большее количество
информации о музыке для пользователей.

Следующие направления дальнейшей работы видятся автору наиболее перспективными.
\begin{enumerate}
  \item Учёт контекста при распознавании аккордов, как локального, так и
  глобального. Дополнительная информация о структуре композиции, о её
  тональности, о жанре, об используемых инструментах может быть полезна в
  некоторых случаях.
  \item Повышение быстродействия существующих алгоритмов. Этому вопросу нечасто
  уделяется внимание в работах, зачастую данные о времени работы алгоритмов не
  приводятся. Однако при использовании алгоритма в реальной жизни вопрос
  быстродействия выходит на первый план, иногда оказываясь более важным, чем
  качество решения задачи алгоритмом. Параллельная обработка звуковых данных,
  эксперименты на больших коллекциях звукозаписей, оптимизация используемых
  алгоритмов -- важные темы с точки зрения применения описанных в работе
  методов.
\end{enumerate}

\clearpage