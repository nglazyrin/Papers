\chapter*{Введение}							% Заголовок
\addcontentsline{toc}{chapter}{Введение}	% Добавляем его в оглавление

% Связь музыки и математики была обнаружена ещё в Античные времена. Пифагор
% заметил, что частоты звуков, составляющих наиболее значимые гармоничные
% интервалы или консонансы, относятся друг к другу как целые числа: 2:1 (октава),
% 3:2 (квинта), 4:3 (кварта) (см. \cite{Christensen2002}, с. 274). Понятия
% окружности и прямой линии позволили представить взаимоотношения мажорных и
% минорных тональностей и консонантных интервалов на плоскости (см.
% \cite{Christensen2002}, с. 283). Модульная арифметика и сложение по модулю 12
% позволили математически выразить эквивалентность нот с одним и тем же названием
% в разных октавах, обращение и транспозицию интервалов. Это отношение
% эквивалентности делит все музыкальные звуки на классы эквивалентности, которые
% называют \emph{высотными классами}. В свою очередь, операции транспозиции и
% обращения задают группу преобразований на множестве высотных классов
% (\cite{Christensen2002}, с. 290). Используя свойства этого множества и его
% булеана, также можно выразить музыкальные взаимоотношения в музыке различных
% эпох и культур. Ещё более абстрактная модель \emph{обобщённой интервальной
% системы} задается тройкой: набор музыкальных объектов, группа обобщённых
% интервалов и функция, сопоставляющая каждой паре объектов обобщённый интервал.
% Такая модель позволяет представить всевозможные музыкальные пространства (см.
% \cite{Christensen2002}, с. 295-296).

Музыка является неотъемлемой составляющей жизни современного человека. Она
проявляет себя в разных формах: от детских колыбельных и напевания под нос до
радио и сигналов вызова сотовых телефонов. Люди таких профессий как музыкант,
музыковед, музыкальный критик, диджей большую часть своей жизни уделяют музыке.
Те, кто не занимается музыкой профессионально, зачастую имеют несколько любимых
исполнителей и слушают музыку время от времени.

На текущий момент компьютер является основным средством для хранения и обработки
музыки и любой информации о музыке, будь то ноты, биография композитора, год
выпуска записи или график концертов группы. Сама по себе музыка, содержащаяся в
цифровых звукозаписях, более ценна для человека, поскольку никакая
информация о ней не может заменить собой её прослушивание. Вместе с тем, именно
эта дополнительная информация даёт возможность ориентироваться в музыкальных
коллекциях, находить новую музыку, организовывать существующие записи. В силу
большей ценности музыки, зачастую звукозаписи не сопровождаются дополнительной
информацией. Необходимость получения разнообразной информации о данной цифровой
звукозаписи порождает множество задач, связанных с обработкой звука:
идентификация композиции, нахождение разных версий одной композиции, определение
заданной композиции в потоке звука с радио, поиск похожих композиций,
определение мелодии композиции для последующего воспроизведения на музыкальном
инструменте и другие. Эта диссертация посвящена задаче определения
последовательности аккордов в звуке.

Аккорды -- это основная информация, необходимая гитаристу для того, чтобы
сыграть композицию. Многочисленные гитаристы-любители, не имеющие достаточно
опыта или усидчивости, чтобы самостоятельно определить звучащие аккорды, смогут
получить инструмент, решающий эту задачу за них. Информация о последовательности
аккордов может быть использована также для индексации композиций и последующего
поиска по запросу. Среди возможных сценариев такого поиска можно отметить
следующие:
\begin{itemize}
  \item поиск заимствований или разных версий одной и той же композиции;
  \item поиск композиций, которые могут гармонично сочетаться друг с другом.
\end{itemize}

\section{Актуальность темы}

Первые попытки обработки музыкальной информации в символьном виде были сделаны в
1950-х годах с появлением первых компьютеров. Они были связаны с автоматическим
определением закономерностей в музыке и использования их для создания новых
мелодий (см. \cite{Schueler2005}). Тогда же предлагается использовать компьютер
для распознавания и печати нотных записей, анализа схожести различных композиций
и поиска по образцу. В 1960-х годах появляются первые работы (например,
\cite{Freedman1967}), связанные с анализом звукозаписей, представленных в
цифровом виде. Их целью было, прежде всего, понимание того, из чего состоят
воспроизводимые музыкальными инструментами звуки и как они воспринимаются
человеком.

В 1975 году в \cite{Moorer1975} было положено начало новому применению
компьютера к анализу цифровых музыкальных звукозаписей: распознаванию в ней
отдельных нот. Этот процесс объединяют с компьютерным распознаванием нотных
записей под общим названием \emph{транскрибирование}. Здесь впервые теория
музыки используется для анализа композиции не в виде нотной записи, а в том
виде, в котором её воспринимает обычный слушатель -- в виде звукозаписи.
Несмотря на раннюю постановку и большое количество приложенных усилий, задача
транскрибирования музыкальной звукозаписи не решена до сих пор.

В 1982 году компаниями Sony и Philips было запущено массовое производство
компакт-дисков, на которых музыка была записана в цифровом формате. Со временем
доступных в цифровом виде произведений стало на порядки больше, чем доступных
нотных записей. Закономерно возрос интерес к автоматическому транскрибированию
музыки. В \cite{Moorer1975} рассматривались только звукозаписи, содержащие не
более двух одновременно звучащих музыкальных инструментов. В 1996 году в
\cite{Martin1996} был представлен один из первых методов, подходивших для любой
полифонической цифровой аудиозаписи.

Задача определения последовательности аккордов при этом не отделялась от задачи
транскрибирования. Как отмечает Т. Фуджишима в \cite{Fujishima1999},
в 1980-1990-х годах (например, в работе \cite{Aono1998}) проблема распознавания
аккордов в музыке решалась путём распознавания отдельных нот и их объединения в
аккорды. Он же впервые предложил метод распознавания аккордов без
предварительного транскрибирования звукозаписи. В \cite{Aono1998} метод
распознавания аккордов являлся частью системы для автоматического
аккомпанемента выступлению живого человека.

В 2000-х годах определение аккордов окончательно выделяется в отдельную задачу.
Начиная с 2008 года в рамках ежегодной кампании по оценке методов музыкального
информационного поиска MIREX
\footnote{\url{http://www.music-ir.org/mirex/wiki/MIREX_HOME}} проводятся
соревнования среди алгоритмов распознавания аккордов в звуке. За это время был
достигнут существенный прогресс в качестве распознавания. В 2012 году на это
соревнование были выставлены более 10 алгоритмов.

В 2010-х годах появляются широко доступные программные продукты, включающие в
себя такие алгоритмы. В популярном пакете для профессионального создания музыки
\emph{Ableton Live 9}\footnote{См. \url{https://www.ableton.com/en/live/}.}
есть возможность записать аккорды или мелодию на гитаре или другом музыкальном
инструменте, после чего преобразовать эту запись в нотное представление в
редакторе. Приложения для смартфонов \emph{AnySong Chord Recognition} \footnote{
\url{https://play.google.com/store/apps/details?id=com.musprojects.chord}} и
\emph{Chord Detector} \footnote{
\url{http://www.chord-detector.com/wordpress/apps/chorddetector/}} позволяют
определить аккорды в звуковом файле и показывают гитарные табулатуры.




(Зачем это надо. MIR и краткая историческая справка с указанием имён,
предшественники.) Разные условия при разных предназначениях. Почему машинное
обучение -- это плохо. Формулировка основной задачи. Краткое содержание работы.
Основные результаты.



Распознавание аккордов больше всего полагается на теорию музыки. Определение
тональности не помогает, т.к. много ошибок - в пределах тональности, соседи на
квинтовом круге (см. 11 - All You Need Is Love). MIREX.
Измерение. (12 Gool Old-Fashioned Lover Boy - пример, когда аккорд длится 1
долю, 10 - Things We Said Today - пропускаем короткие аккорды).

Обзор, введение в тему, обозначение места данной работы в мировых исследованиях и т.п.

\textbf{Целью} данной работы является \ldots

Для~достижения поставленной цели необходимо было решить следующие задачи:
\begin{enumerate}
  \item Исследовать, разработать, вычислить и т.д. и т.п.
  \item Исследовать, разработать, вычислить и т.д. и т.п.
  \item Исследовать, разработать, вычислить и т.д. и т.п.
  \item Исследовать, разработать, вычислить и т.д. и т.п.
\end{enumerate}

\textbf{Основные положения, выносимые на~защиту:}
\begin{enumerate}
  \item Первое положение
  \item Второе положение
  \item Третье положение
  \item Четвертое положение
\end{enumerate}

\textbf{Научная новизна:}
\begin{enumerate}
  \item Впервые \ldots
  \item Впервые \ldots
  \item Было выполнено оригинальное исследование \ldots
\end{enumerate}

\textbf{Научная и практическая значимость} \ldots

\textbf{Степень достоверности} полученных результатов обеспечивается \ldots Результаты находятся в соответствии с результатами, полученными другими авторами.

\textbf{Апробация работы.}
Основные результаты работы докладывались~на:
перечисление основных конференций, симпозиумов и т.п.

\textbf{Личный вклад.} Автор принимал активное участие \ldots

\textbf{Публикации.} Основные результаты по теме диссертации изложены в ХХ печатных изданиях~\cite{bib1,bib2,bib3,bib4,bib5},
Х из которых изданы в журналах, рекомендованных ВАК~\cite{bib1,bib2,bib3}, 
ХХ --- в тезисах докладов~\cite{bib4,bib5}.

\textbf{Объем и структура работы.} Диссертация состоит из~введения, четырех глав, заключения и~двух приложений. Полный объем диссертации составляет ХХХ~страница с~ХХ~рисунками и~ХХ~таблицами. Список литературы содержит ХХХ~наименований.

\clearpage