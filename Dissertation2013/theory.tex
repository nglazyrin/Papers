\chapter{Необходимые теоретические сведения} \label{chaptT}

В этой главе даются теоретические сведения, необходимые для формальной
постановки задачи и описания достигнутых результатов, и обзор существующих
подходов к решению задачи. В разделе \ref{sectT_sound} даются базовые понятия
звука и спектра. В разделе \ref{sectT_prop} описываются основные свойства звука
с точки зрения теории музыки. В разделе \ref{sectT_music} разъясняются основные
понятия из теории музыки, необходимые для дальнейших рассуждений. В разделе
\ref{sectT_digit} представлены основы представления звука в цифровом виде.
В разделе \ref{sectT_musrec} указаны характерные черты музыкальных звукозаписей,
которые могут быть использованы для решения поставленной задачи.

% Необходима вводная часть, аналитическая (основная) часть и выводы. Во вводной
% части обосновывается выбор темы с указанием актуальности и значимости вопроса,
% цель обзора, даются временной интервал и круг источников, тематические границы.

\section{Звук} \label{sectT_sound}

Большая Советская Энциклопедия \cite{Bse1972} определяет звук в широком смысле
как колебательное движение частиц упругой среды, распространяющееся в виде волн в
газообразной, жидкой или твёрдой средах. В воздухе звук передается как
последовательность сгущений и разрежений. Поэтому звук можно считать непрерывной
функцией $x(t)$, показывающей зависимость давления воздуха в данной точке от
времени. В рамках данной работы нас будет интересовать только звук в узком
смысле как явление, субъективно воспринимаемое человеком через органы слуха.
Уловленные ими колебания преобразуются в нервные импульсы, которые передаются в
мозг человека. Воспринимаемый человеком звук $x'(t)$ определяется как общим
строением органов слуха, так и их индивидуальными особенностями для конкретного
человека.

Если звук был вызван колебательным процессом с периодом $T_0$ и частотой $f_0 =
\frac{1}{T_0}$, то полученный звуковой сигнал также будет иметь \emph{период}
$T_0$ и \emph{частоту} $f_0$. Будем называть такой звук \emph{чистым тоном}.
Реальные звуковые сигналы обычно вызваны множеством колебаний с различными
частотами, поэтому можно говорить о \emph{частотном спектре} звука или его
\emph{спектральной функции} $a(f)$. Это неотрицательная функция,
которая показывает зависимость между частотой колебаний и интенсивностью этой
частоты в данном звуковом сигнале $x(t)$. Также выделяют спектр мощности и
фазовый спектр сигнала. В дальнейшем, если не оговорено иное, под спектром будет
пониматься частотный спектр звукового сигнала.

Для любых существующих в природе звуковых сигналов функция $x(t)$ отлична от 0
только на некотором промежутке $[t_{start}, t_{end}]$. Поэтому любой реальный
сигнал можно периодически продолжить на всю вещественную ось с периодом $\tau =
t_{end} - t_{start}$. Более того, продолженная таким образом функция $x(t)$
будет непрерывной и ограниченной. Поэтому она может быть однозначно выражена в
виде ряда гармонических функций (или \emph{гармоник}), частоты которых кратны
$1 / \tau$: $$x(t) = a_0 + \sum_{k=1}^\infty a_k \cos \left(2\pi \frac{k}{\tau}
t - \phi_k \right),$$ где $a_k$ -- амплитуда, а $\phi_k$ -- фаза $k$-й
гармонической функции. Значения $a_k$ составляют спектр звукового сигнала
$x(t)$. Если $x(t)$ является чистым тоном с частотой $f_0$ и периодом $\tau =
1/f_0$, сумма вырождается в одно слагаемое $a_{f_0} \cos(2 \pi f_0 t -
\phi_{f_0})$.

Звуки, издаваемые музыкальными инструментами, не являются чистыми тонами. В
каждом таком звуке можно выделить \emph{основной тон}, имеющий наиболее низкую
частоту, и \emph{обертоны}, имеющие более высокие частоты. Обертоны, у которых
частоты кратны частоте основного тона, называют гармоническими. Они
характерны, например, для струнных музыкальных инструментов. Обертоны с другими
частотами называют негармоническими.

\section{Свойства звука} \label{sectT_prop}

И. В. Способин в \cite{Sposobin2012} выделяет 4 основных свойства звука с точки
зрениия теории музыки: высота, длительность, громкость, тембр. Рассмотрим их
более подробно.

\emph{Высота} звука отражает субъективное восприятие человеком частоты звука.
Высота звука нелинейно (но монотонно) зависит от его частоты. На основе
экспериментов были предложены различные модели этой зависимости, в том числе
шкала мелов и шкала барков. Более подробно эти модели описаны, например, в
\cite{Lerch2012}, с. 79-81. Высота звука может быть выражена с разной степенью
ясности. Высота звуков, имеющих основной тон, определяется его частотой. Для
остальных звуков (например, разного рода шумы, шорохи, в том числе звуки шумящих
музыкальных инструментов) высота может быть неясной.

\emph{Длительность} звука соответствует длительности колебаний источника звука.
Она приобретает особое значение в контексте музыкального произведения, когда
последовательность звуков и их продолжительность задают ритм.

\emph{Громкость} звука определяется амплитудой колебаний. Но, как в случае с
высотой, эта характеристика звука является субъективной. Воспринимаемая
человеком громкость зависит как от амплитуды (нелинейно и монотонно), так и от
высоты звука (нелинейно и немонотонно). Эти зависимости подробно описаны в
\cite{Fastl2007}.

\emph{Тембр} или окраска звука определяется частотами и интенсивностью его
обертонов, которые, в свою очередь, определяются физическими свойствами
музыкального инструмента. Благодаря разнице в тембрах человек может отличать
друг от друга разные музыкальные инструменты.

\section{Основные понятия из теории музыки} \label{sectT_music}

Определения этого раздела даны в соответствии с \cite{Sposobin2012}.

\emph{Музыкальной системой} называется отобранный практикой ряд звуков, которые
находятся в определённых соотношениях по высоте. Музыкальная система является
результатом длительно развивающейся музыкальной практики человеческого общества.
Для нас наиболее привычна система, сформировавшаяся в европейской, в том числе
русской классической музыке. Далее под музыкальной системой будет пониматься
именно эта система.

\emph{Звукорядом} называется совокупность звуков музыкальной системы,
расположенных в порядке высоты (в восходящем или нисходящем направлении).

\emph{Ступенью} называется звук музыкальной системы. Основные ступени
соответствуют звукам, извлекаемым белыми клавишами фортепиано. Им присвоены
собственные названия: \emph{до}, \emph{ре}, \emph{ми}, \emph{фа}, \emph{соль},
\emph{ля}, \emph{си}. Необходимо отдельно отметить, что слово <<нота>>
обозначает графическое изображение звука. Тем не менее, оно часто используется
как синоним для понятия <<звук>>, например, <<нота \emph{до}>> в значении
<<звук \emph{до}>>.

\emph{Строем} называется совокупность постоянных отношений по высоте между
звуками музыкальной системы.

Человек воспринимает звуки с частотами $f_0$ и $2f_0$ как очень похожие и тесно
связанные друг с другом. Расстояние между такими звуками называется
\emph{октавой}. Как отмечает Д. Левитин в \cite{Levitin2006}, <<в основе
музыки каждой из известных нам культур лежит октава [\ldots] даже некоторые
животные -- например, обезьяны и кошки, -- воспринимают звуки, отличающиеся
на октаву, как похожие>>.

Звукоряд делится на октавы на основе октавного сходства его звуков и отражающей
это сходство повторности их названий. В свою очередь, каждая октава имеет своё
название: субконтр-октава, контр-октава, большая октава, малая октава и октавы с
первой по пятую.

\emph{Темперированным} называется строй, который делит каждую октаву звукоряда
на равные части. С начала XVIII века в европейской музыке принята
двенадцатизвуковая (двенадцатиступенная) темперация, делящая октаву на 12 равных
друг другу частей, называемых \emph{полутонами}. Полутон является наименьшим
расстоянием по высоте, возможным в двенадцатизвуковом темперированном строе. Он
образуется между звуками любых двух соседних клавиш на фортепиано. Частоту
каждой ступени звукоряда можно вычислить по формуле
\begin{equation}
\label{eq:note_freq}
f_j = f_0 \cdot 2^{j/12},
\end{equation}
где $f_0$ -- частота настройки музыкальных инструментов. Обычно выбирают $f_0 =
440$ Гц и фиксируют эту частоту для звука \emph{ля} первой октавы. Клавиатура
фортепиано охватывает 88 ступеней: от \emph{ля} субконтроктавы до
ступени \emph{до} пятой октавы. Частота, соответствующая $k$-й слева клавише
фортепиано (отсчитывается с нуля), может быть вычислена по формуле
$$f_k = 27.5 \cdot 2^{k/12}$$
Широко используемый в настоящее время стандарт MIDI, задающий формат обмена
данными между электронными музыкальными инструментами, определяет 128 возможных
значений для частоты звука\footnote{См.
\url{http://www.midi.org/techspecs/midituning.php/}.}. Частота, соответствующая
ступени с номером $k,~0 \leq k \leq 127$, может быть получена по формуле
$$f_k = 2^{\frac{k - 69}{12}} \cdot 440$$
И наоборот, номер ступени может быть получен из частоты по формуле
\begin{equation}
\label{eq:fton}
k = 69 + round \left( 12 \log_2 \left( \frac{f}{440} \right) \right)
\end{equation}
Приведенные выше формулы справедливы для стандартного значения частоты
настройки $f_0 = 440$ Гц. В рамках стандарта MIDI звук \emph{ля} первой октавы
соответствует 69-й ступени.

Производными называются ступени звукоряда, получаемые посредством повышения или
понижения его основных ступеней. Повышение или понижение ступени называется
\emph{альтерацией}. Знаки альтерации указывают на повышение или понижение
основной ступени. Для дальнейшего изложения важны знаки \emph{диез} ($\sharp$) и
\emph{бемоль} ($\flat$), обозначающие соответствено повышение и понижение на
один полутон.

\emph{Интервалом} называется расстояние по высоте между двумя звуками, взятыми
последовательно или одновременно. В октаве заключено 8 ступеней. Соответственно,
имеется 8 основных названий интервалов, отражающих их величину в ступенях.
Каждое название обозначает порядковый номер второго звука интервала, как если бы
от первого его звука брались все ступени до него подряд: прима, секунда, терция,
кварта, квинта, секста, септима, октава. \emph{Обращением} интервала называется
перемещение его нижнего звука на октаву вверх или верхнего звука на октаву вниз.

\emph{Созвучием} называется одновременное сочетание двух и более звуков.
\emph{Аккордом} называется созвучие, состоящее не менее, чем из трёх звуков.
\emph{Гармонией} называется объединение звуков в созвучия и последовательность
созвучий. Гармония формирует контекст, сопровождает мелодию, а также может
объединять несколько одновременно звучащих мелодий. В свою очередь, контекст
формирует у слушателя ожидание последующих событий в музыке. Композитор может
как оправдывать, так и нарушать эти ожидания для большей выразительности.

\emph{Трезвучием} называется аккорд, который состоит из трёх звуков,
располагающихся по терциям. Мажорное трезвучие состоит из большой и малой терций
(4 и 3 полутона соответственно). Минорное трезвучие состоит из малой и большой
терций. Уменьшенное трезвучие состоит из двух малых терций. Увеличенное
трезвучие состоит из двух больших терций. Во всяком трезвучии, независимо от его
типа, нижний звук называется \emph{основным звуком} или \emph{примой}, второй
(по расстоянию от примы) -- \emph{терцией}, а третий -- \emph{квинтой}.
\emph{Основным} аккордом называется такое положение аккорда, в котором основной
звук лежит ниже остальных его звуков. \emph{Обращением} аккорда называется такое
его положение, в котором нижним звуком является терция или квинта основного
трезвучия. Обращения получаются посредством переноса звуков основного трезвучия
вверх на октаву.

\emph{Септаккордом} называется четырехзвучие, располагающееся по терциям.
Септаккорд может быть получен из трезвучия путём добавления к нему одной терции
сверху. Наиболее употребительны доминантсептаккорд (большая, малая, малая
терции), уменьшенный септаккорд (малая, малая, малая терции), малый септаккорд
(малая, малая, большая терции) и минорный септаккорд (малая, большая, малая
терции).

\emph{Ритмом} называется организованная последовательность длительностей звуков.
Основные соотношения звуковых длительностей в музыке таковы, что каждая более
крупная длительность относится к ближайшей более мелкой как 2 к 1. При этом
нотные знаки обозначают только относительную длительность звуков, но не
абсолютную. \emph{Ритмическим рисунком} называется последовательность звуковых
длительностей, взятая отдельно от высотных соотношений звуков.

\emph{Акцентом} называется выделение звука посредством большей громкости (часто
также длительности) по сравнению с окружающими звуками. \emph{Метром} называется
непрерывно повторяющаяся последовательность акцентируемых и неакцентируемых
равнодлительных ритмических единиц (отрезков времени). Акцентируемые и
неакцентируемые равнодлительные ритмические единициы времени, образующие метр,
называются \emph{метрическими долями}. Акцентируемая доля называется
\emph{тяжёлой} или \emph{сильной}, неакцентируемая -- \emph{легкой} или
\emph{слабой}. Акценты, как правило, повторяются через одинаковое количество
долей: через одну, две и т.д.

\emph{Размером} в нотной записи называется метр, доля которого выражена
определённой ритмической длительностью (например, четвертью ноты). Размер
обозначается дробью, числитель которой говорит о количестве его долей, а
знаменатель -- о длительности, которая принята за долю. \emph{Тактом} называется
часть музыкального произведения, которая начинается с тяжёлой доли и
заканчивается перед следующей тяжёлой долей. \emph{Темпом} называется скорость
движения, частота пульсирования метрических долей. Темп иногда указывают числом,
которое обозначает количество ударов метронома в минуту.

Для музыкальной выразительности необходимо объединение нескольких звуков или
созвучий в систему, основанную на определённых высотных соотношениях и связях. В
таких системах есть звуки, используемые как опора (в частности для окончания
мелодии). Эти звуки появляются на тяжёлой доле такта, в конце музыкальной мысли
(что часто бывает на чётных тактах). Кроме того, мелодия время от времени
возвращается к таким звукам. Музыкальная практика выделяет среди таких звуков
один, наиболее устойчивый, который называется \emph{тоникой}. Неустойчивыми
называются звуки системы, в которых выражается незавершённость музыкальной
мысли. \emph{Тяготением} называется притяжение неустойчивого звука системы к
устойчивому, отстоящему от него на секунду. \emph{Ладом} называется система
звуковысотных связей, объединённая тоникой. Многие лады состоят из 7 звуков, но
существуют лады с большим и меньшим их числом. \emph{Тональностью} называется
высотное положение лада. Название тональности состоит из обозначения тоники и
обозначения лада. В двух основных ладах -- мажорном и минорном -- устойчивые
звуки, взятые вместе, образуют соответственно мажорное и минорное трезвучия.

\section{Цифровой звук} \label{sectT_digit}

Звуковой сигнал $x(t)$ может быть представлен в цифровом виде при помощи
операций \emph{дискретизации} и \emph{квантования}. Для этого с некоторой
частотой $\nu$ раз в секунду измеряется амплитуда функции $x(t)$
(дискретизация), после чего каждое полученное значение $x(t_i)$ заменяется на
ближайшее из заданного множества $X_Q$ возможных значений амплитуды
(квантование). Как правило, это множество содержит $2^8$, $2^{16}$ или $2^{24}$
элементов, чтобы каждое значение можно было представить целым числом байт.
Частота $\nu$ часто выбирается равной 44100 Гц (по историческим причинам). При
этом $\nu$ называют \emph{частотой дискретизации}, а значения $x_Q(t_i)$ --
\emph{отсчётами} исходного сигнала $x(t))$. В соответствии с классической
теоремой Котельникова, если спектр сигнала $x(t)$ ограничен сверху частотой
$\nu/2$ (т.е. $a_k = 0$ для $\frac{k}{\tau} > \nu/2$), то исходный сигнал может
быть восстановлен однозначно и без потерь по измеренным значениям $x(t_i)$. При
квантовании эти значения заменяются на $x_Q(t_i)$, поэтому исходный сигнал может
быть восстановлен из оцифрованного только с некоторой ошибкой, которая тем
меньше, чем больше возможных значений амплитуды использовалось при квантовании.
Для большинства звукозаписей эта ошибка незаметна на слух. Отметим ещё раз, что
спектр любых оцифрованных звуковых сигналов ограничен.

\section{Свойства музыкальных звукозаписей} \label{sectT_musrec}

Последовательность аккордов имеет смысл определять в звукозаписи, содержащей
музыку в том или ином виде. Это может быть как студийная запись на
компакт-диске, так и запись гитары через микрофон мобильного телефона.
Музыкальные звукозаписи в целом обладают рядом свойств, которые нужно учитывать
при определении последовательности аккордов. Каждое из них может быть выражено
в большей или меньшей степени или вообще отсутствовать.

\begin{itemize}
  \item Одновременное звучание нескольких музыкальных инструментов. При этом
  звуковые сигналы, издаваемые разными инструментами (и даже разными звучащими
  элементами одного инструмента, например, струнами) складываются. Точно так же
  складываются спектры этих инструментов.
  
  \item Наличие гармоник у музыкальных инструментов с ясно выраженной высотой
  звучания. В звучании таких инструментов можно выделить отдельную ноту. При
  этом наряду с частотой, соответствующей этой основной ноте, звучат другие
  частоты. Их звучание менее выражено, но они могут соответствовать другим
  ступеням музыкальной системы. Математически это означает, что если $k_0$
  таково, что $a_{k_0}$ -- наибольшая из компонент спектра звучащей ноты, то
  существует по меньшей мере одно значение $k > k_0$ такое, что $a_k$
  существенно отлична от 0. Соотношения между парами $(k_0, k)$ для разных $k$ и
  разных $k_0$ (соответствующих разным нотам) во многом задают тембр
  музыкального инструмента.
  
  \item Наличие инструментов с невыраженной высотой звучания. К ним относятся
  многие ударные инструменты, в звучании которых невозможно выделить конкретную
  ноту. Спектр таких инструментов характеризуется большим количеством
  расположенных подряд существенно отличных от нуля значений, слабо отличающихся
  друг от друга. Иными словами, существуют такие положительные числа $A$ и
  $\delta$, что $A$ существенно больше $\delta$ и $A < a_k < A + \delta$ для
  всех $k$ из некоторого промежутка $[k_0, k_1]$.
  
  \item Наличие ритма и метра. Сильные метрические доли обычно акцентируются
  ударными инструментами и началом звучания нот. Слабые метрические доли часто
  выделяются ударными инструментами. Широко употребляются простые метры, где
  акцент делается один раз на две или три доли. Также широкоупотребителен метр,
  состоящий из четырёх долей с акцентом на первой и третьей, при этом акцент на
  первой доле чуть сильнее. Другие метры и размеры используются реже.
  
  \item Наличие лада и тональности. Они позволяют объединить в целостную
  композицию звуки различных музыкальных инструментов и голоса. Они накладывают
  ограничения на допустимые аккорды в композиции. Вместе с тем, эти ограничения
  не являются строгими и могут сознательно нарушаться композиторами. Кроме того,
  тональность может меняться на протяжении композиции, что влечёт за собой
  изменение набора <<допустимых>> аккордов.
  
  \item Наличие повторений. Как пишет Д. Левитин в \cite{Levitin2006},
  <<музыка основана на повторениях>>. Одна и та же музыкальная фраза,
  последовательность аккордов и даже целый фрагмент композиции могут повториться
  в точности или с небольшими изменениями.
\end{itemize}

\section{Формализация задачи} \label{sectT_task}

Пусть заданы звуковой сигнал $x(t),~t \in [t_{start}, t_{end}]$ и множество
возможных названий аккордов $Y$. Необходимо для каждого момента времени $t \in
[t_{start}, t_{end}]$ указать аккорд $y \in Y$, звучащий в этот момент.

Разобьем задачу на отдельные этапы.

\subsection{Частотно-временное представление}

Представление звука в виде последовательности отсчётов амплитуды не является
удобным для обработки: неясно, как сопоставить аккорду последовательность
отсчётов и наоборот. Поэтому естественным первым шагом является часто
используемый при обработке звука переход к частотно-временному представлению
звукозаписи, или получению её спектрограммы $C_{N \times M}$. Основным
инструментом для такого перехода является дискретное оконное преобразование
Фурье.

Спектрограмма представляет из себя матрицу, каждая из $N$ строк которой
соответствует определённой частоте, а каждый из $M$ столбцов -- промежутку
времени. Элементами её являются значения интенсивности данной частоты на данном
промежутке времени. Фактически, каждый столбец представляет из себя спектр
короткого фрагмента исходного сигнала. Удобно представлять спектрограмму в виде
последовательности столбцов $C_{N \times M} = \{C_m\}_{m=0}^{M-1}$.

Здесь возникает 2 подзадачи: разбиение звукозаписи на фрагменты и вычисление
спектра на каждом из них. Важно отметить, что для современной западной музыки
характерны использование равномерно темперированного строя и наличие ритма.
Поэтому большее количество звуковой энергии должно быть сосредоточено в точках,
соответствующих частотам нот и моментам начала метрических долей. Исходя из этих
соображений, можно скорректировать выбор моментов начала фрагментов и выбор
используемого преобразования. Удобно использовать одно и то же преобразование на
всех фрагментах, поскольку в этом случае все столбцы спектрограммы будут иметь
одинаковый размер и смысл. После этого на каждом фрагменте необходимо решить
задачу классификации.

\subsection{Классификация}

Обозначим за $C \subset \mathbb{R}^N$ множество всех возможных векторов-столбцов
спектрограммы. Для каждого вектора из данной последовательности
$\{C_m\}_{m=0}^{M-1}$ необходимо указать аккорд $y \in Y$, соответствующий
этому вектору. Принимая во внимание известные из равномерно темперированного
строя частоты нот, можно по спектру звука на данном фрагменте делать
предположения относительно звучащего аккорда. Возможно использование не только
текущего, но и предыдущих векторов (а также последующих, если от алгоритма не
требуется выдавать результат в реальном времени). Также возможно использование
результатов классификации других векторов.

На этом этапе важными вопросами являются выбор метода классификации и выбор
целевой функции (если метод предполагает нахождение наилучших параметров путём
обучения). Основной подзадачей является отыскание такого набора преобразований
множества $X$, который позволит уменьшить количество ошибок классификации с
использованием выбранного метода. Результатом преобразования будет
последовательность векторов признаков, отличная от исходной последовательности
векторов-столбцов спектрограммы. Для отыскания подходящих преобразований могут
быть использованы свойства, перечисленные в разделе \ref{sectT_musrec}.

В главе \ref{chaptL} описываются основные походы, применяемые для решения
каждой из отмеченных выше подзадач.

\clearpage