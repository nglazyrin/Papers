\chapter{Необходимые теоретические сведения} \label{chaptT}

В этой главе даются теоретические сведения, необходимые для формальной
постановки задачи и описания достигнутых результатов, и обзор существующих
подходов к решению задачи. В разделе \ref{sectT_sound} даются базовые понятия
звука и спектра. В разделе \ref{sectT_prop} описываются основные свойства звука
с точки зрения теории музыки. В разделе \ref{sectT_music} разъясняются основные
понятия из теории музыки, необходимые для дальнейших рассуждений. В разделе
\ref{sectT_digit} представлены основы представления звука в цифровом виде.
В разделе \ref{sectT_musrec} указаны характерные черты музыкальных звукозаписей,
которые могут быть использованы для решения поставленной задачи.

Раздел \ref{sectT_lit} посвящён обзору литературы в области распознавания
аккордов. Практически все затронутые работы вышли не позднее 15 лет с момента
написания данной диссертации, наиболее значимые результаты были получены в
течение последних 5 лет.

% Необходима вводная часть, аналитическая (основная) часть и выводы. Во вводной
% части обосновывается выбор темы с указанием актуальности и значимости вопроса,
% цель обзора, даются временной интервал и круг источников, тематические границы.

\section{Звук} \label{sectT_sound}

Большая Советская Энциклопедия определяет звук в широком смысле как
колебательное движение частиц упругой среды, распространяющееся в виде волн в
газообразной, жидкой или твёрдой средах. В воздухе звук передается как
последовательность сгущений и разрежений. Поэтому звук можно считать непрерывной
функцией $x(t)$, показывающей зависимость давления воздуха в данной точке от
времени. В рамках данной работы нас будет интересовать только звук в узком
смысле как явление, субъективно воспринимаемое человеком через органы слуха.
Уловленные ими колебания преобразуются в нервные импульсы, которые передаются в
мозг человека. Воспринимаемый человеком звук $x'(t)$ определяется как общим
строением органов слуха, так и их индивидуальными особенностями для конкретного
человека.

Если звук был вызван колебательным процессом с периодом $T_0$ и частотой $f_0 =
\frac{1}{T_0}$, то полученный звуковой сигнал также будет иметь \emph{период}
$T_0$ и \emph{частоту} $f_0$. Будем называть такой звук \emph{чистым тоном}.
Реальные звуковые сигналы обычно вызваны множеством колебаний с различными
частотами, поэтому можно говорить о \emph{частотном спектре} звука или его
\emph{спектральной функции} $a(f)$. Это неотрицательная функция,
которая показывает зависимость между частотой колебаний и интенсивностью этой
частоты в данном звуковом сигнале $x(t)$. Также выделяют спектр мощности и
фазовый спектр сигнала. В дальнейшем, если не оговорено иное, под спектром будет
пониматься частотный спектр звукового сигнала.

Для любых существующих в природе звуковых сигналов функция $x(t)$ отлична от 0
только на некотором промежутке $[t_0, t_1]$. Поэтому любой реальный сигнал можно
периодически продолжить на всю вещественную ось с периодом $\tau = t_1 - t_0$.
Более того, продолженная таким образом функция $x(t)$ будет непрерывной и
ограниченной. Поэтому она может быть однозначно выражена в виде ряда
гармонических функций (или \emph{гармоник}), частоты которых кратны $1 / \tau$:
$$x(t) = a_0 + \sum_{k=1}^\infty a_k \cos \left(2\pi \frac{k}{\tau} t -
\phi_k \right),$$ где $a_k$ -- амплитуда, а $\phi_k$ -- фаза $k$-й гармонической
функции. Значения $a_k$ составляют спектр звукового сигнала $x(t)$. Если $x(t)$
является чистым тоном с частотой $f_0$ и периодом $\tau = 1/f_0$, сумма
вырождается в одно слагаемое $a_{f_0} \cos(2 \pi f_0 t - \phi_{f_0})$.

Звуки, издаваемые музыкальными инструментами, не являются чистыми тонами. В
каждом таком звуке можно выделить \emph{основной тон}, имеющий наиболее низкую
частоту, и \emph{обертоны}, имеющие более высокие частоты. Обертоны, у которых
частоты кратны частоте основного тона, называют гармоническими. Они
характерны, например, для струнных музыкальных инструментов. Обертоны с другими
частотами называют негармоническими.

\section{Свойства звука} \label{sectT_prop}

И. В. Способин в \cite{Sposobin2012} выделяет 4 основных свойства звука с точки
зрениия теории музыки: высота, длительность, громкость, тембр. Рассмотрим их
более подробно.

\emph{Высота} звука отражает субъективное восприятие человеком частоты звука.
Высота звука нелинейно (но монотонно) зависит от его частоты. На основе
экспериментов были предложены различные модели этой зависимости, в том числе
шкала мелов и шкала барков. Более подробно эти модели описаны, например, в
\cite{Lerch2012}, с. 79-81. Высота звука может быть выражена с разной степенью
ясности. Высота звуков, имеющих основной тон, определяется его частотой. Для
остальных звуков (например, разного рода шумы, шорохи, в том числе звуки шумящих
музыкальных инструментов) высота может быть неясной.

\emph{Длительность} звука соответствует длительности колебаний источника звука.
Она приобретает особое значение в контексте музыкального произведения, когда
последовательность звуков и их продолжительность задают ритм.

\emph{Громкость} звука определяется амплитудой колебаний. Но, как в случае с
высотой, эта характеристика звука является субъективной. Воспринимаемая
человеком громкость зависит как от амплитуды (нелинейно и монотонно), так и от
высоты звука (нелинейно и немонотонно). Эти зависимости подробно описаны в
\cite{Fastl2007}.

\emph{Тембр} или окраска звука определяется частотами и интенсивностью его
обертонов, которые, в свою очередь, определяются физическими свойствами
музыкального инструмента. Благодаря разнице в тембрах человек может отличать
друг от друга разные музыкальные инструменты.

\section{Основные понятия из теории музыки} \label{sectT_music}

Определения этого раздела даны в соответствии с \cite{Sposobin2012}.

\emph{Музыкальной системой} называется отобранный практикой ряд звуков, которые
находятся в определённых соотношениях по высоте. Музыкальная система является
результатом длительно развивающейся музыкальной практики человеческого общества.
Для нас наиболее привычна система, сформировавшаяся в европейской, в том числе
русской классической музыке. Далее под музыкальной системой будет пониматься
именно эта система.

\emph{Звукорядом} называется совокупность звуков музыкальной системы,
расположенных в порядке высоты (в восходящем или нисходящем направлении).

\emph{Ступенью} называется звук музыкальной системы. Основные ступени
соответствуют звукам, извлекаемым белыми клавишами фортепиано. Им присвоены
собственные названия: \emph{до}, \emph{ре}, \emph{ми}, \emph{фа}, \emph{соль},
\emph{ля}, \emph{си}. Необходимо отдельно отметить, что слово <<нота>>
обозначает графическое изображение звука. Тем не менее, оно часто используется
как синоним для понятия <<звук>>, например, <<нота \emph{до}>> в значении
<<звук \emph{до}>>.

\emph{Строем} называется совокупность постоянных отношений по высоте между
звуками музыкальной системы.

Человек воспринимает звуки с частотами $f_0$ и $2f_0$ как очень похожие и тесно
связанные друг с другом. Расстояние между такими звуками называется
\emph{октавой}. Как отмечает Д. Левитин в \cite{Levitin2006}, <<в основе
музыки каждой из известных нам культур лежит октава [\ldots] даже некоторые
животные -- например, обезьяны и кошки, -- воспринимают звуки, отличающиеся
на октаву, как похожие>>.

Звукоряд делится на октавы на основе октавного сходства его звуков и отражающей
это сходство повторности их названий. В свою очередь, каждая октава имеет своё
название: субконтр-октава, контр-октава, большая октава, малая октава и октавы с
первой по пятую.

\emph{Темперированным} называется строй, который делит каждую октаву звукоряда
на равные части. С начала XVIII века в европейской музыке принята
двенадцатизвуковая (двенадцатиступенная) темперация, делящая октаву на 12 равных
друг другу частей, называемых \emph{полутонами}. Полутон является наименьшим
расстоянием по высоте, возможным в двенадцатизвуковом темперированном строе. Он
образуется между звуками любых двух соседних клавиш на фортепиано. Частоту
каждой ступени звукоряда можно вычислить по формуле
\begin{equation}
\label{eq:note_freq}
f_i = f_0 \cdot 2^{i/12},
\end{equation}
где $f_0$ -- частота настройки музыкальных инструментов. Обычно выбирают $f_0 =
440$ Гц и фиксируют эту частоту для звука \emph{ля} первой октавы. Клавиатура
фортепиано охватывает 88 ступеней: от \emph{ля} субконтроктавы до
ступени \emph{до} пятой октавы. Частота, соответствующая $n$-й слева клавише
фортепиано (отсчитывается с нуля), может быть вычислена по формуле
$$f_n = 27.5 \cdot 2^{n/12}$$
Широко используемый в настоящее время стандарт MIDI, задающий формат обмена
данными между электронными музыкальными инструментами, определяет 128 возможных
значений для частоты звука\footnote{См.
\url{http://www.midi.org/techspecs/midituning.php/}.}. Частота, соответствующая
ступени с номером $n,~0 \leq n \leq 127$, может быть получена по формуле
$$f_n = 2^{\frac{n-69}{12}} \cdot 440$$
И наоборот, номер ступени может быть получен из частоты по формуле
\begin{equation}
\label{eq:fton}
n = 69 + 12 \log_2 \left( \frac{f}{440} \right)
\end{equation}
Приведенные выше формулы справедливы для стандартного значения частоты
настройки $f_0 = 440$ Гц. В рамках стандарта MIDI звук \emph{ля} первой октавы
соответствует 69-й ступени.

Производными называются ступени звукоряда, получаемые посредством повышения или
понижения его основных ступеней. Повышение или понижение ступени называется
\emph{альтерацией}. Знаки альтерации указывают на повышение или понижение
основной ступени. Для дальнейшего изложения важны знаки \emph{диез} ($\sharp$) и
\emph{бемоль} ($\flat$), обозначающие соответствено повышение и понижение на
один полутон.

\emph{Интервалом} называется расстояние по высоте между двумя звуками, взятыми
последовательно или одновременно. В октаве заключено 8 ступеней. Соответственно,
имеется 8 основных названий интервалов, отражающих их величину в ступенях.
Каждое название обозначает порядковый номер второго звука интервала, как если бы
от первого его звука брались все ступени до него подряд: прима, секунда, терция,
кварта, квинта, секста, септима, октава. \emph{Обращением} интервала называется
перемещение его нижнего звука на октаву вверх или верхнего звука на октаву вниз.

\emph{Созвучием} называется одновременное сочетание двух и более звуков.
\emph{Аккордом} называется созвучие, состоящее не менее, чем из трёх звуков.
\emph{Гармонией} называется объединение звуков в созвучия и последовательность
созвучий. Гармония формирует контекст, сопровождает мелодию, а также может
объединять несколько одновременно звучащих мелодий. В свою очередь, контекст
формирует у слушателя ожидание последующих событий в музыке. Композитор может
как оправдывать, так и нарушать эти ожидания для большей выразительности.

\emph{Трезвучием} называется аккорд, который состоит из трёх звуков,
располагающихся по терциям. Мажорное трезвучие состоит из большой и малой терций
(4 и 3 полутона соответственно). Минорное трезвучие состоит из малой и большой
терций. Уменьшенное трезвучие состоит из двух малых терций. Увеличенное
трезвучие состоит из двух больших терций. Во всяком трезвучии, независимо от его
типа, нижний звук называется \emph{основным звуком} или \emph{примой}, второй
(по расстоянию от примы) -- \emph{терцией}, а третий -- \emph{квинтой}.
\emph{Основным} аккордом называется такое положение аккорда, в котором основной
звук лежит ниже остальных его звуков. \emph{Обращением} аккорда называется такое
его положение, в котором нижним звуком является терция или квинта основного
трезвучия. Обращения получаются посредством переноса звуков основного трезвучия
вверх на октаву.

\emph{Септаккордом} называется четырехзвучие, располагающееся по терциям.
Септаккорд может быть получен из трезвучия путём добавления к нему одной терции
сверху. Наиболее употребительны доминантсептаккорд (большая, малая, малая
терции), уменьшенный септаккорд (малая, малая, малая терции), малый септаккорд
(малая, малая, большая терции) и минорный септаккорд (малая, большая, малая
терции).

\emph{Ритмом} называется организованная последовательность длительностей звуков.
Основные соотношения звуковых длительностей в музыке таковы, что каждая более
крупная длительность относится к ближайшей более мелкой как 2 к 1. При этом
нотные знаки обозначают только относительную длительность звуков, но не
абсолютную. \emph{Ритмическим рисунком} называется последовательность звуковых
длительностей, взятая отдельно от высотных соотношений звуков.

\emph{Акцентом} называется выделение звука посредством большей громкости (часто
также длительности) по сравнению с окружающими звуками. \emph{Метром} называется
непрерывно повторяющаяся последовательность акцентируемых и неакцентируемых
равнодлительных ритмических единиц (отрезков времени). Акцентируемые и
неакцентируемые равнодлительные ритмические единициы времени, образующие метр,
называются \emph{метрическими долями}. Акцентируемая доля называется
\emph{тяжёлой} или \emph{сильной}, неакцентируемая -- \emph{легкой} или
\emph{слабой}. Акценты, как правило, повторяются через одинаковое количество
долей: через одну, две и т.д.

\emph{Размером} в нотной записи называется метр, доля которого выражена
определённой ритмической длительностью (например, четвертью ноты). Размер
обозначается дробью, числитель которой говорит о количестве его долей, а
знаменатель -- о длительности, которая принята за долю. \emph{Тактом} называется
часть музыкального произведения, которая начинается с тяжёлой доли и
заканчивается перед следующей тяжёлой долей. \emph{Темпом} называется скорость
движения, частота пульсирования метрических долей. Темп иногда указывают числом,
которое обозначает количество ударов метронома в минуту.

Для музыкальной выразительности необходимо объединение нескольких звуков или
созвучий в систему, основанную на определённых высотных соотношениях и связях. В
таких системах есть звуки, используемые как опора (в частности для окончания
мелодии). Эти звуки появляются на тяжёлой доле такта, в конце музыкальной мысли
(что часто бывает на чётных тактах). Кроме того, мелодия время от времени
возвращается к таким звукам. Музыкальная практика выделяет среди таких звуков
один, наиболее устойчивый, который называется \emph{тоникой}. Неустойчивыми
называются звуки системы, в которых выражается незавершённость музыкальной
мысли. \emph{Тяготением} называется притяжение неустойчивого звука системы к
устойчивому, отстоящему от него на секунду. \emph{Ладом} называется система
звуковысотных связей, объединённая тоникой. Многие лады состоят из 7 звуков, но
существуют лады с большим и меньшим их числом. \emph{Тональностью} называется
высотное положение лада. Название тональности состоит из обозначения тоники и
обозначения лада. В двух основных ладах -- мажорном и минорном -- устойчивые
звуки, взятые вместе, образуют соответственно мажорное и минорное трезвучия.

\section{Цифровой звук} \label{sectT_digit}

Звуковой сигнал $x(t)$ может быть представлен в цифровом виде при помощи
операций \emph{дискретизации} и \emph{квантования}. Для этого с некоторой
частотой $\nu$ раз в секунду измеряется амплитуда функции $x(t)$
(дискретизация), после чего каждое полученное значение $x(t_i)$ заменяется на
ближайшее из заданного множества $X_Q$ возможных значений амплитуды
(квантование). Как правило, это множество содержит $2^8$, $2^{16}$ или $2^{24}$
элементов, чтобы каждое значение можно было представить целым числом байт.
Частота $\nu$ часто выбирается равной 44100 Гц (по историческим причинам). При
этом $\nu$ называют \emph{частотой дискретизации}, а значения $x_Q(t_i)$ --
\emph{отсчётами} исходного сигнала $x(t))$. В соответствии с классической
теоремой Котельникова, если спектр сигнала $x(t)$ ограничен сверху частотой
$\nu/2$ (т.е. $a_k = 0$ для $\frac{k}{\tau} > \nu/2$), то исходный сигнал может
быть восстановлен однозначно и без потерь по измеренным значениям $x(t_i)$. При
квантовании эти значения заменяются на $x_Q(t_i)$, поэтому исходный сигнал может
быть восстановлен из оцифрованного только с некоторой ошибкой, которая тем
меньше, чем больше возможных значений амплитуды использовалось при квантовании.
Для большинства звукозаписей эта ошибка незаметна на слух. Отметим ещё раз, что
спектр любых оцифрованных звуковых сигналов ограничен.

\section{Свойства музыкальных звукозаписей} \label{sectT_musrec}

Последовательность аккордов имеет смысл определять в звукозаписи, содержащей
музыку в том или ином виде. Это может быть как студийная запись на
компакт-диске, так и запись гитары через микрофон мобильного телефона.
Музыкальные звукозаписи в целом обладают рядом свойств, которые нужно учитывать
при определении последовательности аккордов. Каждое из них может быть выражено
в большей или меньшей степени или вообще отсутствовать.

\begin{itemize}
  \item Одновременное звучание нескольких музыкальных инструментов. При этом
  звуковые сигналы, издаваемые разными инструментами (и даже разными звучащими
  элементами одного инструмента, например, струнами) складываются. Точно так же
  складываются спектры этих инструментов.
  
  \item Наличие гармоник у музыкальных инструментов с ясно выраженной высотой
  звучания. В звучании таких инструментов можно выделить отдельную ноту. При
  этом наряду с частотой, соответствующей этой основной ноте, звучат другие
  частоты. Их звучание менее выражено, но они могут соответствовать другим
  ступеням музыкальной системы. Математически это означает, что если $k_0$
  таково, что $a(k_0)$ -- наибольшая из компонент спектра звучащей ноты, то
  существует по меньшей мере одно значение $k > k_0$ такое, что $a(k)$
  существенно отлична от 0. Соотношения между парами $(k_0, k)$ для разных $k$ и
  разных $k_0$ (соответствующих разным нотам) во многом задают тембр
  музыкального инструмента.
  
  \item Наличие инструментов с невыраженной высотой звучания. К ним относятся
  многие ударные инструменты, в звучании которых невозможно выделить конкретную
  ноту. Спектр таких инструментов характеризуется большим количеством
  расположенных подряд существенно отличных от нуля значений, слабо отличающихся
  друг от друга. Иными словами, существуют такие положительные числа $L$ и
  $\delta$, что $L$ существенно больше $\delta$ и $L < a(k) < L + \delta$ для
  всех $k$ из некоторого промежутка $[k_0, k_1]$.
  
  \item Наличие ритма и метра. Сильные метрические доли обычно акцентируются
  ударными инструментами и началом звучания нот. Слабые метрические доли часто
  выделяются ударными инструментами. Широко употребляются простые метры, где
  акцент делается один раз на две или три доли. Также широкоупотребителен метр,
  состоящий из четырёх долей с акцентом на первой и третьей, при этом акцент на
  первой доле чуть сильнее. Другие метры и размеры используются реже.
  
  \item Наличие лада и тональности. Они позволяют объединить в целостную
  композицию звуки различных музыкальных инструментов и голоса. Они накладывают
  ограничения на допустимые аккорды в композиции. Вместе с тем, эти ограничения
  не являются строгими и могут сознательно нарушаться композиторами. Кроме того,
  тональность может меняться на протяжении композиции, что влечёт за собой
  изменение набора <<допустимых>> аккордов.
  
  \item Наличие повторений. Как пишет Д. Левитин в \cite{Levitin2006},
  <<музыка основана на повторениях>>. Одна и та же музыкальная фраза,
  последовательность аккордов и даже целый фрагмент композиции могут повториться
  в точности или с небольшими изменениями.
\end{itemize}

\section{Формализация задачи} \label{sectT_task}

Пусть заданы звуковой сигнал $x(t),~t \in [t_{start}, t_{end}]$ и множество
возможных названий аккордов $Y$. Необходимо для каждого момента времени $t \in
[t_{start}, t_{end}]$ указать аккорд $y \in Y$, звучащий в этот момент. Из такой
формулировки не ясен способ решения данной задачи. Разобьем задачу на отдельные
этапы.

\subsection{Частотно-временное представление}

Представление звука в виде последовательности отсчётов амплитуды не является
удобным для обработки: неясно, как сопоставить аккорду последовательность
отсчётов и наоборот. Поэтому естественным первым шагом является часто
используемый при обработке звука переход к частотно-временному представлению
звукозаписи, или получению её спектрограммы $C = C_{M \times N}$. Основным
инструментом для такого перехода является дискретное оконное преобразование
Фурье.

Спектрограмма представляет из себя матрицу, каждая из $M$ строк которой
соответствует определённой частоте, а каждый из $N$ столбцов -- промежутку
времени. Элементами её являются значения интенсивности данной частоты на данном
промежутке времени. Фактически, каждый столбец представляет из себя спектр
короткого фрагмента исходного сигнала. Удобно представлять спектрограмму в виде
набора столбцов $C = (C_0, C_1, \ldots, C_{N-1})$.

Здесь возникает 2 подзадачи: разбиение звукозаписи на фрагменты и вычисление
спектра на каждом из них. Важно отметить, что для современной западной музыки
характерны использование равномерно темперированного строя и наличие ритма.
Поэтому большее количество звуковой энергии должно быть сосредоточено в точках,
соответствующих частотам нот и моментам начала метрических долей. Исходя из этих
соображений, можно скорректировать выбор моментов начала фрагментов и выбор
используемого преобразования. Удобно использовать одно и то же преобразование на
всех фрагментах, поскольку в этом случае все столбцы спектрограммы будут иметь
одинаковый размер и смысл. После этого на каждом фрагменте необходимо решить
задачу классификации.

\subsection{Классификация}

Обозначим за $X \subset \mathbb{R}^M$ множество всех возможных векторов-столбцов
спектрограммы. Для каждого вектора из данной последовательности $C = (C_0, C_1,
\ldots, C_{N-1}),~C_i \in X,~i=\overline{0,N-1}$ необходимо указать аккорд $y
\in Y$, соответствующий этому вектору. Принимая во внимание известные из
равномерно темперированного строя частоты нот, можно по спектру звука на данном
фрагменте делать предположения относительно звучащего аккорда. Возможно
использование не только текущего, но и предыдущих векторов (а также
последующих, если от алгоритма не требуется выдавать результат в реальном
времени). Также возможно использование результатов классификации других
векторов.

На этом этапе важными вопросами являются выбор метода классификации и выбор
целевой функции (если метод предполагает нахождение наилучших параметров путём
обучения). Основной подзадачей является отыскание такого набора преобразований
множества $X$, который позволит уменьшить количество ошибок классификации с
использованием выбранного метода. Результатом преобразования будет
последовательность векторов признаков, отличная от исходной последовательности
векторов-столбцов спектрограммы. Для отыскания подходящих преобразований могут
быть использованы свойства, перечисленные в разделе \ref{sectT_musrec}.

В разделе \ref{sectT_lit} описываются основные походы, применяемые для решения
каждой из отмеченных выше подзадач.

\section{Обзор литературы} \label{sectT_lit}

Подраздел \ref{ssectT_prelim} описывает существующие подходы к разбиению
звукозаписи на фрагменты, а также некоторые другие преобразования, совершаемые
над звукозаписью перед её дальнейшей обработкой. Подраздел \ref{ssectT_spect}
посвящён существующим подходам к получению спектрограммы. В подразделе
\ref{ssectT_feat} рассмотрены существующие методы преобразования спектрограммы,
позволяющие более точно классифицировать аккорд. Наконец, подраздел
\ref{ssectT_post} описывает часто применяемые методы классификации.

% \subsection{Общая схема алгоритмов распознавания аккордов} \label{ssectT_scheme}
% 
% Описанные в разделе \ref{sectT_musrec} свойства дают дополнительную информацию,
% которая успешно используется в различных алгоритмах распознавания аккордов.
% Такие алгоритмы можно представить в виде серии преобразований, каждое из которых
% использует одно или несколько из перечисленных свойств музыкальных звукозаписей.
% Эту серию можно условно разделить на три этапа, второй из которых обязательно
% присутствует в любом алгоритме.

% \subsubsection{Предварительная обработка} \label{sssectT_prelim}

% \subsubsection{Спектрограмма и последовательность векторов признаков}
% \label{sssectT_feat}

% Спектрограмма показывает, как меняется распределение звуковой энергии по
% частотам со временем. Фактически, это матрица, состоящая из неотрицательных
% действительных чисел. Удобно представлять её в виде набора столбцов $C = (C_0,
% C_1, \ldots, C_M)$, каждый из которых является спектром короткого фрагмента
% исходной звукозаписи. Если на предыдущем этапе были определены моменты начала
% метрических долей, их можно использовать для разделения композиции на фрагменты.
% Если была определена также частота настройки, её можно использовать для выбора
% частот для компонент спектрограммы. Можно предполагать, что подобранные таким
% образом компоненты будут оказываться в те моменты времени и на тех частотах, где
% сосредоточено наибольшее количество звуковой энергии.

% Над спектрограммой производятся преобразования, нацеленные на подавление
% инструментов с неясной высотой звучания и выделение компонент, соответствующих
% звучащим нотам. Также на этом этапе учитывается, что человек воспринимает звуки
% с частотами, отличающимися на октаву, как похожие. Компоненты спектра,
% соответствующие одной и той же ноте в разных октавах, комбинируются. Результатом
% всех преобразований является последовательность векторов признаков, в которой
% каждый вектор соответствует одному столбцу спектрограммы, но имеет при этом
% существенно меньшую размерность.
  
% \subsubsection{Обработка последовательности векторов признаков}
% \label{sssectT_post}
% Последовательность векторов признаков можно напрямую преобразовывать в
% последовательность аккордов, классифицируя каждый вектор как соответствующий
% тому или иному аккорду. Тем не менее, можно улучшить качество распознавания,
% используя самые высокоуровневые свойства музыкальных звукозаписей: наличие лада,
% тональности, повторов. Накладывая ограничения на допустимые последовательности и
% сочетания аккордов, они увеличивают вероятность правильно классифицировать
% вектор признаков.

\subsection{Предварительная обработка} \label{ssectT_prelim}

На этом этапе собирается информация, которая будет использоваться для получения
частотно-временного представления звукозаписи: определяются моменты начала
метрических долей и частота настройки музыкальных инструментов. Иногда
дополнительно производится разделение звука на гармонические и перкуссионные
компоненты, после чего последние удаляются из сигнала. К этому же этапу можно
отнести понижение частоты дискретизации цифровой звукозаписи для ускорения
вычислений на следующем этапе и преобразование стереофонических записей в
монофонические.

Понижение частоты дискретизации применяется во многих работах
(\cite{Sheh2003}, \cite{Bello2005}, \cite{Lee2006}, \cite{Burgoyne2007},
\cite{Lee2007}, \cite{Papadopoulos2007}, \cite{Mauch2008}, \cite{Khadkevich2009},
\cite{Mauch2009}, \cite{Oudre2009}, \cite{Reed2009}, \cite{Mauch2010},
\cite{Khadkevich2011}, \cite{Ni2011}, \cite{Humphrey2012}) для ускорения
обработки файла. Как правило, частота дискретизации понижается с часто
используемой 44100 Гц до 11025 Гц путем замены каждых 4 подряд идущих
отсчётов $x_Q(t_i), x_Q(t_{i+1}), x_Q(t_{i+2}), x_Q(t_{i+3})$ на $x_Q(t_i)$.
Вместе с тем, такое преобразование не является общепринятым. В \cite{Zhang2008},
\cite{Cho2010}, \cite{Rocher2010}, \cite{Cho2011}, \cite{DeHaas2012} частота
дискретизации звукозаписей не меняется. Общепринятым является преобразование
стерофонических звукозаписей в монофонические путём взятия среднего
арифметического от сигналов левого и правого каналов.

Определение моментов начала метрических долей позволяет на следующем шаге
получить спектрограмму, соотносящуюся с ритмом композиции. Смена аккорда, как и
любое другое событие в музыке, очень часто подчинена ритму и происходит на
границе метрических долей. Кроме того, в столбцах спектрограммы, соответствующих
акцентированным долям, будет более ярко выражено звучание инструментов и
соответствующие им пики спектра. Моменты начала метрических долей используются
как для деления звука на фрагменты, каждый из которых соответствует одной доле
или её части (\cite{Yoshioka2004}, \cite{Sumi2008}, \cite{Weller2009},
\cite{Mcvicar2011}, \cite{Ni2011}), так и для усреднения столбцов спектрограммы,
вычисленной с фиксированным шагом по времени, в пределах одной метрической доли
(\cite{Bello2005}, \cite{Mauch2009}, \cite{Mauch2010}, \cite{DeHaas2012},
\cite{Chen2012}. В \cite{Chen2012} исследовались оба этих варианта, а также
дополнительно медианная фильтрация по всем столбцам спектрограммы в пределах
одной метрической доли. Наилучший результат был получен в случае усреднения
столбцов спектрограммы в пределах метрической доли. В остальных случаях авторы
произволно выбирали один из вариантов использования этой информации.

Отслеживание ритма (beat detection) является одной из популярных задач
музыкального информационного поиска. Новые алгоритмы появляются каждый год, но
далеко не все из них применются при распознавании аккордов. Практически все
алгоритмы распознавания аккордов используют один из алгоритмов, представленных в
\cite{Davies2007}, \cite{Dixon2007}, \cite{Ellis2007}. Соответствующие
программные модули для этих алгоритмов доступны бесплатно и удобны в
подключении, что, по-видимому, является основной причиной их использования.

Отклонение частоты настройки музыкальных инструментов от стандартного значения
440 Гц может как явно определяться на данном этапе (\cite{Gomez2006},
\cite{Papadopoulos2007}, \cite{Khadkevich2009}, \cite{Khadkevich2011},
\cite{Ni2011}, \cite{Jiang2011}, так и неявно учитываться в процессе обработки
спектрограммы (\cite{Bello2005}, \cite{Lee2006}, \cite{Reed2009},
\cite{Mauch2010}, \cite{Rocher2010}). Основные алгоритмы для определения частоты
настройки представлены в работах \cite{Harte2005}, \cite{Zhu2005},
\cite{Gomez2006}, \cite{Peeters2006}, \cite{KhadkevichPhase2009}.

Разделение звука на гармонические и перкуссионные компоненты позволяет ослабить
влияние музыкальных инструментов с неясной высотой звучания на спектрограмму,
получаемую на следующем этапе. Аналогичные преобразования делаются на этапе
преобразования спектрограммы в последовательность векторов признаков во многих
работах. Но в \cite{Reed2009} и \cite{Ni2011} перкуссионные компоненты удаляются
из сигнала до построения спектрограммы при помощи свободно доступной для
научного использования реализации алгоритма \cite{Ono2008}.

\subsection{Спектрограмма} \label{ssectT_spect}

Переход к частотно-временному представлению звукозаписи в виде спектрограммы
является ключевым, поскольку даёт возможность работать с отдельными частотными
компонентами звука. Как было отмечено выше, для этого звукозапись делится на
короткие, возможно, пересекающиеся фрагменты, на каждом из которых вычисляется
спектр звука.

В алгоритмах распознавания аккордов используются следующие методы получения
спектра.
\begin{enumerate}
  \item Дискретное оконное преобразование Фурье.
  $$X[k] = \sum_{k=0}^{N-1} w(n)x_Q(t_n) e^{-\frac{i 2\pi kn}{N}}, \quad
  k=0, 1, \ldots, K$$
  Здесь $N$ -- размер анализируемого фрагмента звукозаписи в отсчётах, $w(n)$ --
  функция, отличная от нуля на некотором промежутке, не выходящем за пределы
  этого фрагмента -- оконная функция. Прямоугольная оконная функция $w(n)$,
  равная 1 только на анализируемом фрагменте и 0 -- вне его, получается
  автоматически при разделении на фрагменты исходной звукозаписи. Среди других
  оконных функций наиболее популярной является окно Хемминга:
  $$w(n) = 0.53836 - 0.46164 \cos \left( \frac{2\pi n}{N-1} \right)$$
  При использовании оконной функции результатом преобразования Фурье является не
  спектр исходного сигнала, а спектр его произведения с оконной функцией.  
  Согласно свойству преобразования Фурье, этот спектр будет равен свёртке
  спектров исходного сигнала и оконной функции. Её выбор влияет на форму
  полученных искажений спектра. Более подробную информацию об эффектах от выбора
  оконной функции можно найти в \cite{Oppenheim2006}, раздел 10.3.1.
  
  Достоинствами дискретного оконного преобразования Фурье являются существование
  быстрых алгоритмов вычисления в определённых случаях и наличие большого
  количества реализаций на разных языках программирования. Вместе с тем, при
  использовании алгоритмов быстрого преобразования Фурье невозможно произвольным
  образом выбирать частоты его компонент. Это создает неудобства при дальнейшей
  обработке, поскольку невозможно точно определить количество звуковой энергии,
  приходящейся на частоты, соответствующие ступеням звукоряда. Дискретное
  оконное преобразование Фурье используется в \cite{Sheh2003},
  \cite{Gomez2006}, \cite{Burgoyne2007}, \cite{Papadopoulos2007},
  \cite{Khadkevich2009}, \cite{Weller2009}, \cite{Mauch2010},
  \cite{Khadkevich2011}, \cite{DeHaas2012}.
  
  \item Преобразование постоянного качества (constant Q преобразование).
  $$X[k] = \frac{1}{N(k)} \sum_{k=0}^{N(k)-1} w(k,n)x_Q(t_n) e^{-\frac{i 2\pi
  kn}{N(k)}}, \quad k=0, 1, \ldots, K$$ Здесь, в отличие от преобразования
  Фурье, размер анализируемого фрагмента и размер оконной функции зависят от
  номера соответствующей частотной компоненты $f_k$. В свою очередь, $f_k$ можно
  выбрать таким образом, что каждой ступени звукоряда будет соответствовать
  одинаковое число частотных компонент (одна или более). Пусть $b$ -- количество
  компонент в одной октаве, а $f_{min}$ -- частота наименьшей из анализируемых
  компонент. Тогда частота $k$-й компоненты задается формулой $f_k = 2^{k/b}
  f_{min}$. Точно так же задаются частоты для ступеней звукоряда при
  использовании равномерно темперированного строя, поэтому параметр $f_{min}$
  напрямую связан с частотой настройки музыкальных инструментов. Отношение
  $\frac{f_k}{f_{k+1} - f_k} = \frac{1}{2^{1/b}-1} = Q$ называется коэффициентом
  качества. При таком выборе частот $Q$ не зависит от $k$. Отсюда происходит
  название constant-Q преобразования.
  
  Достоинством этого преобразования является легкость дальнейшей работы со
  спектром, поскольку его компоненты напрямую соответствуют ступеням звукоряда.
  Недостатками являются большая сложность вычислений и зависимость от
  правильного определения частоты настройки. Более быстрый алгоритм вычисления
  преобразования постоянного качества, использующий результат быстрого
  преобразования Фурье исходного сигнала, был предложен в \cite{Brown1992}.
  Преобразование постоянного качества используется в \cite{Bello2005},
  \cite{Lee2006}, \cite{Mauch2008}, \cite{Mauch2009}, \cite{Oudre2009},
  \cite{Reed2009}, \cite{Cho2010}, \cite{Cho2011}, \cite{Ni2011}.
  
  \item Гребёнка фильтров (filter bank). В цифровой обработке сигналов любое
  преобразование сигнала называют фильтром. Известно (см. \cite{Rabiner1978}, с.
  424-425), что быстрое преобразование Фурье эквивалентно вполне определенной
  гребёнке достаточно грубых фильтров. Вместо них можно использовать любые
  другие фильтры, у каждого из которых центр полосы пропускания соответствует
  частоте одной из ступеней звукоряда, а ширина полосы пропускания достаточно
  мала, чтобы не охватывать частоты соседних ступеней. Эти фильтры можно
  подобрать так, чтобы они были менее грубыми, то есть более точно определяли
  количество звуковой энергии, приходящейся на их полосы пропускания. Также
  можно выбирать полосы пропускания фильтров в соответствии с частотами ступеней
  звукоряда. Недостатком данного метода является большая вычислительная
  сложность в сравнении с алгоритмом быстрого преобразования Фурье. Гребёнки
  фильтров используются в \cite{Jiang2011}, \cite{Humphrey2012}.
\end{enumerate}

\subsection{Векторы признаков} \label{ssectT_feat}

Переход от столбцов спектрограммы к векторам признаков основан на том, что
человек воспринимает звуки с частотами, отличающимися на октаву, как похожие.
Эта же особенность используется композиторами, когда инструменты, звучащие в
разных частотных полосах, воспроизводят одну и ту же ноту в разных октавах, или
несколько голосов из разных октав составляют один аккорд. Поэтому вполне
естественно просуммировать в каждом столбце спектрограммы компоненты,
соответствующие одному и тому же звуку в разных октавах. Пусть спектрограмма
была получена в результате преобразования постоянного качества, и $b$ --
количество частотных компонент в одной октаве в столбце $C_i$, что
соответствует шагу в $b/12$ полутонов. Ко всем значениям $C_i[j], ~ 0 \leq j <
b$, прибавляются значения $C_i[j+b], C_i[j+2b], C_i[j+3b], \ldots$ для каждого
$0 \leq i \leq M$. В результате из последовательности столбцов $\{C_i\}_{i=0}^M$
получается последовательность $b$-мерных векторов $\{B_i\}_{i=0}^M$.

Если при получении спектрограммы использовалось быстрое преобразование Фурье с
размером фрагмента $N$ отсчётов, то необходимо сопоставить компоненты спектра
частотам звукоряда:
\begin{equation}
B_i[j] = \sum_{k: p[k]=j} ||C_i[k]||^2 \label{fft_wrap}
\end{equation}
$$p[k] = \left(round \left[ b~\log_2 \left( \frac{k}{N} \cdot \frac{f_s}{f_0}
\right) \right] \right) \mod b$$
Эта формула применима и для спектрограммы, полученной преобразованием
постоянного качества.

Векторы признаков, полученные путём объединения спектральной информации по всем
октавам, носят общее название векторов хроматических признаков или
\emph{хроматических векторов}. Впервые такой процесс был предложен в
\cite{Fujishima1999}, а соответствующие признаки получили название
\emph{профиль тональных классов} (pitch class profile). Под \emph{тональным
классом} здесь понимается совокупность звуков, имеющих одно название, но
находящихся в разных октавах, например, все звуки \emph{до}.

В отличие от столбцов исходной спектрограммы, каждый хроматический вектор имеет
всего 12 компонент, а значит, соответствующая задача классификации решается в
пространстве меньшей размерности. Каждая из координатных осей в этом
пространстве соответствует уровню энергии, приходящемуся на один тональный
класс. Недостатками такого преобразования является потеря информации об октавах
исходных звуков (влекущая потерю информации об обращении аккорда) и наложение
шумовых компонент спектра на полезные. Несмотря на это, хроматические векторы
используются в большинстве существующих алгоритмов распознавания аккордов.
Для преодоления второго недостатка существует множество дополнительных
преобразований.

В \cite{Lee2006} было предложено для случая спектрограммы, полученной быстрым
преобразованием Фурье, перед вычислением (\ref{fft_wrap}) заменять каждое
значение $C_i[j]$ на $\prod_{m=0}^M |X(2^m \cdot j)|$, где $M$ -- параметр,
регулирующий количество гармоник. Это позволяет учесть информацию о гармониках
инструментов с определённой высотой звучания в векторе признаков, который был
назван \emph{расширенный профиль тональных классов} (enhanced pitch class
profile).

В \cite{Gomez2006} было предложено для случая спектрограммы, полученной быстрым
преобразованием Фурье, учитывать только спектральные пики (локальные максимуы в
каждом столбце). Каждый из них учитывался при вычислении не одного, а
нескольких компонентов вектора, с разными весами, в зависимости от разницы
между частотой пика и частотой ступени звукоряда. Кроме того, чтобы учесть
наличие гармоник, каждый пик с частотой $f_i$ прибавлялся к пикам с частотами
$f_i, f_i/2, f_i/3, \ldots$ с соответствующими весами. Такой вектор признаков
получил название \emph{гармонический профиль тональных классов} (harmonic
pitch class profile).

В \cite{Weller2009} и \cite{Khadkevich2011} были предложены способы
перераспределения звуковой энергии в пределах спектрограммы, полученной быстрым
преобразованием Фурье, от участков с меньшим количеством энергии к участкам с
большим количеством энергии (эта техника была предложена в \cite{Kodera1978}). В
\cite{Weller2009} допускается только перемещение энергии в пределах одного
стоблца спектрограммы, в \cite{Khadkevich2011} допускается таже перемещение
энергии между столбцами. В полученных таким образом спектрограммах более чётко
выделены горизонтальные участки с большим количеством звуковой энергии,
соответствующие инструментам с определённой высотой звучания и их гармоникам.

В \cite{Mauch2010} каждый столбец $C_i$ спектрограммы, полученной быстрым
преобразованием Фурье, преобразуется аналогично (\ref{fft_wrap}) в вектор $Y_i$
из 256 компонент, расположенных с шагом в $1/3$ полутона, что соответствует
охвату в чуть более, чем 7 октав. После этого для 84 ступеней звукоряда от
\emph{ля} субконтроктавы (27.5 Гц) до \emph{фа} третьей октавы (3322 Гц)
генерируются шаблонные 256-компонентные векторы-столбцы. В каждом из них
элементы, соответствующие ступени звукоряда и её гармоникам, задаются как
$s^{k-1}$, где $s=0.6$, а $k$ -- номер гармоники; остальные элементы равны 0.
Взятые вместе, они образуют матрицу $E$. Далее линейным методом наименьших
квадратов находится вектор $x_i$, минимизирующий $||Y_i - Ex_i||$, при условии,
что все компоненты $x_i$ неотрицательны. Полученные векторы $x_i$ образуют новую
спектрограмму с шагом по частоте в $1/3$ полутона, которая обрабатывается как
если бы она была получена в результате преобразования постоянного качества.
Полученный в результате хроматический вектор признаков получил название
\emph{NNLS chroma} (Non-Negative Least Squares).

Как показано в \cite{Fletcher1933}, воспринимаемая громкость звука
приблизительно пропорциональна десятичному логарифму уровня звуковой мощности
(sound power level). Мощность звука определяется как энергия, передаваемая
звуковой волной через рассматриваемую поверхность в единицу времени. Спектр
мощности звука показывает изменение его мощности с течением времени. Он может
быть получен из частотного спектра путем возведения в квадрат каждой из его
компонент. Поэтому имеет смысл перед преобразованием спектрограммы в
последовательность хроматических векторов заменить каждое её значение $C_i[j]$
на $\log (\eta \cdot C_i[j] + 1)$, где $\eta$ -- положительная константа,
которая обычно выбирается из диапазона $100 \leq \eta \leq 10000$. Тогда
соотношение между разными компонентами спектрограммы будет приблизительно
соответствовать соотношению между воспринимаемыми человеком уровнями громкости
соответствующих частот. Полученные таким образом признаки называют
\emph{chroma-log-pitch} (CLP) \cite{Jiang2011}.

В \cite{Mueller2009} было предложено после логарифмирования элементов
спектрограммы для каждого столбца $C_i$ вычислять дискретное косинусное
преобразование, занулять первые $\xi$ полученных коэффициентов, после чего
выполнять обратное дискретное косинусное преобразование. Похожие действия
выполняются при вычислении мел-частотных кепстральных коэффициентов
\cite{Logan2000}, широко используемых в распознавании речи. Из полученной
спектрограммы обычным образом вычисляются хроматические векторы. Они получили
название \emph{chroma DCT-reduced log pitch} (CRP). Целью этого преобразования
является повышение устойчивости хроматических векторов к изменению тембра
музыкальных инструментов, прежде всего для сопоставления различных музыкальных
записей. Но CRP-признаки были успешно применены к распознаванию аккордов в
\cite{Cho2011}.

В \cite{Ni2011} было предложено наряду с зависимостью человеческого восприятия
громкости от звуковой мощности учитывать зависимость от частоты звука. Для этого
на каждом фрагменте звукозаписи вместо частотного спектра вычисляется спектр
мощности, от каждой его компоненты вычисляется десятичный логарифм, после чего к
каждой компоненте применяется A-взвешивание \cite{TalbotSmith1999}.

В \cite{Mueller2007} были предложены преобразования последовательности
хроматических векторов, направленные на повышение устойчивости к шумам.
В последовательности полученных обычным способом хроматических векторов каждый
вектор $B_i$ заменяется на $B_i/||B_i||_1$, где $||B_i||_1 = \sum_{j=0}^b-1
|B_i[j]|$. Затем производится квантование значений $B_i[j],~0 \leq B_i[j] \leq
1$ с порогами, величины которых расположены логарифмически. Далее вычисляется
свёртка последовательности $\{B_i\}_{i=0}^M$ с окном Ханна длины $w \in
\mathbb{N}$, а затем прореживание полученной последовательности по основанию
$d$. Полученные в результате этих преобразований векторы признаков получили
название \emph{chroma energy normalized statictics} ($CENS_d^w$).

В \cite{Mauch2008}, \cite{Mauch2009}, \cite{Khadkevich2011}, \cite{Ni2011},
\cite{DeHaas2012}, \cite{Chen2012} строятся отдельные спектрограммы для
низкочастотной и высокочастотной области спектра, граница между которыми обычно
пролегает в диапазоне от 200 Гц до 250 Гц. Соответственно, получается два
набора хроматических векторов, используемых в дальнейшем анализе.

В \cite{Harte2006} были предложены особые признаки, не являющиеся
хроматическими. Они являются векторами в пространстве \emph{Tonnetz}
\cite{Cohn1998}, \cite{Chew2000}, моделирующем взаимоотношения между ступенями
равномерно темперированного строя. Согласно \cite{Harte2006}, в случае
равномерно темперированного строя это 6-мерное пространство. Для удобства
векторы в этом пространстве нормируют так, чтобы они попадали внутрь 6-мерного
эллипса с радиусами $(r_1, r_1, r_2, r_2, r_3, r_3)$. Координаты можно разделить
попарно на 3 круга. Первый из них в некотором роде соответствует квинтовому
кругу. В нём точки, соответствующие ступеням звукоряда, расположены на
окружности радиуса $r_1$ с шагом $5\pi / 6$. Во втором круге эти точки
расположены на окружности радиуса $r_2$ с шагом $\pi/4$, а в третьем -- на
окружности радиуса $r_3$ с шагом $\pi/3$. Их можно мыслить как круги малых и
больших терций соответственно. Точка, соответствующая аккорду, имеет координаты,
равные среднему арифметическому координат составляющих его нот. Любой
хроматический вектор может быть легко преобразован в вектор в этом пространстве.
Такие векторы признаков были использованы в \cite{Lee2007}, \cite{Lee2008},
\cite{Chen2012}, \cite{Humphrey2012}.

Принципиально другой подход к получению вектора признаков был предложен в
\cite{Humphrey2012}. Описанные выше 6-мерные признаки получаются из
спектрограммы путём применения свёрточной нейронной сети \cite{LeCun1998}. При
этом не применяются никакие знания о свойствах спектра или музыки.
Предполагается, что нейронная сеть сама определит наиболее характерные свойства
в процессе обучения.

Сравнение качества работы некоторых из описанных типов признаков в приложении к
задаче распознавания аккордов было проведено в \cite{Jiang2011}.

\subsection{Классификация векторов признаков}
\label{ssectT_post}

На этом этапе находится решение задачи распознавания аккордов в звукозаписи:
последовательность векторов признаков преобразуется в последовательность
аккордов с указанием моментов начала и конца их звучания. Перед вычислением
спектрограммы звукозапись была поделена на фрагменты, моменты начала и конца
которых известны. Поэтому считается, что каждый из полученных векторов признаков
соответствует промежутку времени между началами текущего и следующего
фрагментов.

Для определения звучащего на данном фрагменте аккорда по вектору признаков
необходимо классифицировать этот вектор. В рамках задачи MIREX Audio Chord
Estimation 2012 выделялось 25 возможных классов: по одному классу для каждого
мажорного и минорного аккордов, а также один класс для отсутствия аккорда.
Многие алгоритмы также ограничиваются этим набором (\cite{Bello2005},
\cite{Lee2006}, \cite{Khadkevich2009}, \cite{Oudre2009}, \cite{Weller2009},
\cite{Cho2010}, \cite{Rocher2010}, \cite{Cho2011}, \cite{Jiang2011},
\cite{Ni2011}, \cite{Chen2012}, \cite{Humphrey2012}). В некоторых работах
выделяют также отдельные классы для доминантсептаккордов (\cite{Sheh2003},
\cite{Mauch2008}, \cite{Zhang2008}, \cite{Mauch2009}, \cite{Mauch2010},
\cite{DeHaas2012}), других септаккордов (\cite{Sheh2003}, \cite{Mauch2010}),
уменьшенных и увеличенных (\cite{Sheh2003}, \cite{Burgoyne2007},
\cite{Lee2008}, \cite{Mauch2008}, \cite{Sumi2008}, \cite{Mauch2009},
\cite{Mauch2010}, \cite{Ni2012}) и других видов аккордов.

\subsubsection{Метод ближайшего соседа}

Наиболее простой способ классификации -- определение расстояния от вектора
признаков до <<идеальных>> шаблонных векторов той же размерности,
соответствующих аккордам. В качестве результата выбирается аккорд, расстояние до
шаблона которого является наименьшим. Фактически, это метод $k$ ближайших
соседей для $k=1$. Такой подход был применён в \cite{Lee2006}, \cite{Oudre2009},
в одном из вариантов \cite{Jiang2011}. Мерой расстояния может выступать
косинусное расстояние, евклидово расстояние, расхождение Кульбака-Лейблера и
другие. Их сравнение было проведено в \cite{Oudre2009}. В качестве шаблона
аккордов часто используют вектор, у которого на позициях, соответствующих
входящим в аккорд нотам, стоят 1, а на остальных -- 0. Например, шаблон для
аккорда до-минор имеет вид $(1,0,0,1,0,0,0,1,0,0,0,0)$ (при условии, что первая
компонента вектора соответствует звуку \emph{до}).

Важным достоинством такого способа классификации является отсутствие этапа
обучения (см. раздел \ref{sect_weak}). Отсюда следует лёгкость добавления новых
типов распознаваемых аккордов: для этого требуется всего лишь добавить новые
шаблоны. Недостатком является невозможность учесть зависимость между подряд
идущими фрагментами звукозаписи.

Иногда (например, в \cite{Oudre2009}) в шаблоны также включают информацию о
гармонических обертонах входящих в аккорд нот. Ноты, соответствующие частотам
гармонических обертонов, могут быть получены из формулы (\ref{eq:fton}). Вклад
обертона в соответствующую компоненту шаблона определялся в \cite{Gomez2006} как
\begin{equation} \label{eq:templates_harmonics}
w_{harm}(n) = s^{n-1}
\end{equation}
где $n$ -- номер обертона, а $s < 1$ -- параметр. Соответствующий шаблонный
вектор будет иметь компоненты со значениями, отличными от 0 и 1.

Для повышения устойчивости к шумам к последовательности векторов признаков
можно предварительно применить скользящий медианный фильтр или фильтр
скользящего среднего, как в \cite{Lee2006}, \cite{Oudre2009}. 

\subsubsection{Скрытые марковские модели и байесовские сети}

\emph{Скрытые марковские модели} (СММ) \cite{Rabiner1989} стали основой многих
алгоритмов распознавания аккордов. В отличие от алгоритма ближайшего соседа, они
позволяют в явном виде моделировать вероятность перехода между двумя заданными
аккордами. Дадим формальное определение элементов СММ.

\begin{itemize}

\item Набор состояний модели $Q = \{Q_1, Q_2, \ldots , Q_N\}$. За
$q_t$ будем обозначать состояние модели в момент времени $t$.

\item Множество наблюдаемых символов $\Lambda = \{\lambda_1, \lambda_2, ...,
\lambda_M\}$.

\item Матрица переходных вероятностей $\Omega = \{\omega_{ij}\}$, где
$\omega_{ij} = P(q_t = Q_j | q_{t-1} = Q_i), \! 1 \leq i,j \leq N$. Если любое
состояние достижимо из любого, то все $\omega_{ij}$ неотрицательны. Для всех
$i,~1 \leq i \leq N$ верно $\sum_{j=1}^N \omega_{ij} = 1$

\item Распределение вероятностей появления наблюдаемых символов в состоянии
$Q_j$, $V=\{v_j(k)\}$, где $v_j(k) = P\{\lambda_k \: at \: t|q_t = S_j\}$ при $1
\leq j \leq N, \: 1 \leq k \leq M$.

\item Начальное распределение вероятностей состояний $\pi =
\{\pi_i\}$, где $\pi_i = P\{q_t = Q_i\}, \: 1 \leq i \leq N$.

\end{itemize}

Состояния СММ ненаблюдаемы, в каждый момент времени доступен для наблюдения
только какой-либо символ из множества $\Lambda$. Важным свойством СММ является
то, что вероятность перехода из состояния $Q_i$ в состояние $Q_j$ не зависит от
предыдущих состояний модели.

Набор состояний СММ фиксируется заранее. В качестве наблюдаемых символов обычно
выступают векторы признаков. Матрица переходных вероятностей, параметры
распределения вероятностей появления наблюдаемых символов и параметры начального
распределения вероятностей состояний могут как задаваться изначально (как в
\cite{Bello2005}, в одном из вариантов \cite{Papadopoulos2007}, в нескольких
вариантах \cite{Cho2010}), так и определяться в результате обучения СММ (как в
\cite{Burgoyne2007}, в нескольких вариантах \cite{Papadopoulos2007},
 \cite{Mauch2008}, \cite{Khadkevich2009}, \cite{Reed2009}, в одном из вариантов
\cite{Cho2010}, \cite{Jiang2011}, \cite{Khadkevich2011}, \cite{Ni2011}).
Вероятности появления наблюдаемых символов обычно моделируются одним многомерным
нормальным распределением (как в \cite{Sheh2003}, \cite{Bello2005},
\cite{Papadopoulos2007}, \cite{Ni2011}, \cite{Chen2012}) или смесью многомерных
нормальных распределений (как в \cite{Burgoyne2007}, \cite{Khadkevich2009},
\cite{Reed2009}, \cite{Cho2010}, \cite{Khadkevich2011}). При обучении обычно
используется итеративный метод математического ожидания -- модификации
(expectation-modification), также называемый методом Баума-Уэлша или методом
прямого-обратного хода. В \cite{Reed2009} минимизируется ошибка классификации,
параметры модели обновляются при помощи градиентного спуска. При распознавании
наиболее вероятной последовательности скрытых состояний применяется алгоритм
Витерби. Стоит отметить, что иногда алгоритм Витерби применяют, не вводя явно
СММ, а задавая псевдовероятности вместо необходимых в алгоритме распределений
вероятностей (например, в \cite{Cho2011}, \cite{Humphrey2012}).

Если наблюдаемыми символами являются хроматические векторы, и вероятности
появления наблюдаемых символов моделируются многомерными распределениями с
числом измерений, равным размерности хроматического вектора, то возможна
дополнительная коррекция параметров СММ после обучения. В хроматическом векторе
каждая компонента соответствует одному классу звуков, например, всем звукам
\emph{до}. Если его первую компоненту такого вектора, соответствущую классу
звуков \emph{до}, переставить в конец, то полученный вектор останется
хроматическим, но его первая компонента будет соответствовать классу звуков
\emph{до-диез}. Аналогичные циклические перестановки возможны для математических
ожиданий и матрицы ковариации соответствующего многомерного распределения.

Так, если в векторе математических ожиданий для распределения, соответствующего
аккорду \emph{до-диез-мажор}, переставить одну компоненту из начала в конец, то
полученный вектор математических ожиданий будет соответствовать аккорду
\emph{до-мажор}. Такими сдвигами можно привести все векторы матожиданий для
распределений, соответствующих мажорным аккордам, к виду, в котором компонента,
соответствующая основному звуку аккорда, будет первой. После этого можно
усреднить все математические ожидания по всем аккордам, и обратными сдвигами
вернуть усреднённые векторы матожиданий на свои места. Аналогично можно
усреднить матрицы ковариации для всех аккордов одного типа. Также возможно
усреднение компонентов матрицы переходов для случаев переходов между аккордами
соответствующих типов, основные звуки которых отстоят на одинаковое число
полутонов. Процедура усреднения применяется, например, в \cite{Sheh2003},
\cite{Papadopoulos2007}, \cite{Cho2010}, \cite{Khadkevich2011}. Усреднение
параметров модели полезно в случае недостатка обучающих данных или их
неравномерного распределения по рассматриваемому набору аккордов.

СММ была впервые применена к определению последовательности аккордов в
\cite{Sheh2003}. Каждое из её скрытых состояний соответствует одному аккорду, и
возможны переходы из любого состояния в любое. Аналогичный подоход к построению
СММ применялся в \cite{Bello2005}, \cite{Lee2007}. Другой подход состоит в
построении отдельной модели для каждого аккорда и связывании этих моделей в
одну СММ с общими входом и выходом для каждой из моделей. Он применялся в одном
из вариантов \cite{Burgoyne2007}, \cite{Mauch2008}, \cite{Khadkevich2009},
\cite{Khadkevich2011}.

Очевидно, что смена аккорда производится не при каждой смене звукового
фрагмента. Поэтому отдельной проблемой является моделирование длительности
нахождения модели в одном состоянии. В случае СММ первого типа на это можно
влиять, регулируя значения на главной диагонали матрицы переходов. В случае СММ
второго типа можно регулировать параметры моделей каждого отдельного аккорда
(как в \cite{Mauch2008}), а также добавлять штраф за переход от модели одного
аккорда к модели другого аккорда (как в \cite{Khadkevich2009},
\cite{Khadkevich2011}). В \cite{Chen2012} в СММ было дополнительно введено
распределение, задающее вероятность нахождения модели в состоянии $Q_i$ в
течение $d$ фрагментов, где $d \leq D=20$. Процедура обучения и алгоритм Витерби
были соответствующим образом модифицированы.

Описанные выше варианты СММ предполагают только зависимость аккорда в текущем
фрагменте от аккорда в предыдущем фрагменте. В \cite{Khadkevich2009} было
предложено использовать языковую модель, которая позволяет учитывать более чем
одно предыдущее состояние СММ.

Вместе с тем, были сделаны попытки учесть при построении модели тональность и
жанр музыки. В \cite{Lee2007} было предложено строить 24 СММ, по одной для
каждой из мажорных и минорных тональностей. При распознавании аккордов для
каждой модели определялась наиболее вероятная последовательность состояний. В
качестве результата выбиралась та из последовательностей, вероятность которой
была наибольшей. Дополнительным результатом при этом было определение
тональности композиции. В \cite{Lee2008} аналогичным образом строились отдельные
СММ для 6 различных музыкальных жанров. В \cite{Ni2012} отдельные СММ строились
для 11 различных жанров, но при этом они были объединены в одну гипер-жанровую
модель с более сложной процедурой обучения. Несмотря на большой потенциал такого
рода комбинаций, они требуют существенно больше обучающих данных. В случае
\cite{Lee2007} и \cite{Lee2008} использовались звукозаписи, сгенерированные из
MIDI-файлов. В \cite{Ni2012} исползовался достаточно большой набор реальных
музыкальных звукозаписей.

Ещё одним способом улучшить производительность СММ является добавление в неё
12 скрытых состояний для текущей басовой ноты и 24 скрытых состояний для текущей
тональности \cite{Ni2011}. При этом общее количество комбинаций скрытых
состояний становится слишком большим, поэтому приходится накладывать
дополнительные ограничения на допустимые переходы между аккордами и между
тональностями и на допустимые сочетания аккордов и басовых нот.

Несмотря на большую популярность, способности СММ к моделированию музыкальных
взаимоотношений ограничены. В большинстве вариантов они моделируют только
зависимость аккорда на данном фрагменте от аккорда на предыдущем фрагменте. Для
введения в такую модель понятий тональности и жанра приходится создавать
достаточно сложные конструкции. В \cite{Mauch2010} было предложено использовать
динамическую байесовскую сеть, которая, по сути, является обобщением СММ (см.
\cite{Ghahramani2001}). В ней используются скрытые состояния для текущих
метрической позиции, тональности, аккорда и басовой ноты; наблюдениями являются
2 вектора хроматических признаков: для высоких и для низких частот. Такая модель
позволяет моделировать сложные музыкальные взаимоотношения. С другой стороны,
она имеет множество параметров, и поэтому требует большего количества обучающих
данных. Для получения наиболее вероятной последовательности в такой сети можно
использовать модификацию алгоритма Витерби, но из-за размеров сети этот процесс
оказывается более длительным, чем в случае СММ.

\subsubsection{Другие модели}

В \cite{Weller2009} было предложено использовать более сильный алгоритм
классификации, чем метод ближайшего соседа, основанный на методе опорных
векторов. Помимо текущего вектора признаков этот алгоритм позволяет учитывать
также признаки на предыдущем или на следующем фрагменте звукозаписи, а также
попарные произведения компонент вектора признаков.

В одном из вариантов \cite{Burgoyne2007} было предложено заменить СММ на
условное случайное поле \cite{Lafferty2001}. Оно определяется следующим образом.
Обозначим за $\boldsymbol{X}$ и $\boldsymbol{Y}$ множество наблюдений и
множество случайных переменных соответственно. Пусть $G = (V, E)$ -- такой граф,
что $\boldsymbol{Y} = (\boldsymbol{Y}_v)_{v \in V}$, то есть $\boldsymbol{Y}$
можно проиндексировать вершинами этого графа. Тогда $(\boldsymbol{X},
\boldsymbol{Y})$ называется \emph{условным случайным полем}, если случайные
переменные $\boldsymbol{Y}_v$ при условии $\boldsymbol{X}$ удовлетворяют
марковскому свойству с учётом графа: $p(\boldsymbol{Y}_v | \boldsymbol{X},
\boldsymbol{Y}_w, w \sim v) = p(\boldsymbol{Y}_v | \boldsymbol{X},
\boldsymbol{Y}_w, w \sim v)$, где $w \sim v$ означает, что $w$ и $v$ являются
соседями в графе $G$. В случае, когда $G$ является цепью или деревом, к
соответствующему условному случайному полю можно применять алгоритмы,
аналогичные методу прямого-обратного хода и алгоритму Витерби. В отличие от СММ,
при определении наиболее вероятной последовательности вершин графа
максимизируется не $p(\boldsymbol{X}, \boldsymbol{Y})$, а $p(\boldsymbol{Y} |
\boldsymbol{X})$. Кроме того, в такой модели каждое скрытое состояние зависит не
только от текущего наблюдения, но от всей предыдущей последовательности
наблюдений. В \cite{Burgoyne2007} отмечается, что условное случайное поле
обучается существенно дольше, чем СММ.

В \cite{DeHaas2012} была предложена полноценная модель гармонии, построенная на
основе музыкально-теоретических соотношений между аккордами. Её применение
требует знания тональности, поэтому для звукозаписи предварительно определяется
последовательность тональностей с ограничением на минимальную длину фрагмента в
одной тональности в 16 метрических долей. На каждом фрагменте звука определяется
набор наиболее вероятных аккордов (вычисляются расстояния от хроматического
вектора до шаблонов аккордов), после чего модель гармонии используется для
определения наиболее вероятной последовательности аккордов с учётом уже
определённых аккордов на всех предыдущих фрагментах.

В \cite{Yoshioka2004} использовался собственный алгоритм для определения
вероятности гипотез. Каждая гипотеза состоит из последовательности аккордов,
определённой до данного фрагмента, и тональности. На каждом фрагменте
определяется вероятность гипотез со всеми возможными вариантами текущего
аккорда. В формуле для вычисления вероятности гипотезы учитываются тональность,
вероятность смены аккорда, хроматический вектор, басовый звук, сочетаемость
аккорда и басового звука. Очень похожий подход с другими формулами для
определения вероятности гипотез был применён в \cite{Sumi2008}.

Подход, в чём-то похожий на алгоритм Витерби, был предложен в \cite{Rocher2010}.
Здесь на каждом фрагменте определяется набор наиболее вероятных аккордов
(вычисляются расстояния от хроматического вектора до шаблонов аккордов) и
тональностей (вычисляются расстояния от хроматического вектора до шаблонных
векторов тональностей из \cite{Temperley2001}). Затем все наиболее вероятные
кандидаты объединяются в пары. Расстояние между парами (аккорд, тональность)
определяется в соответствии с \cite{Lerdahl2001} на основе взаимоотношений
между звуками, составляющими аккорды, и звуками, входящими в тональности. Тогда
методом динамического программирования можно определить последовательность пар
(аккорд, тональность) по всем фрагментам, имеющую наименьшую сумму расстояний
между соседними парами.

В \cite{Mauch2009} было предложено учитывать структуру композиции перед
определением аккордов. Структура может быть задана заранее или определена
автоматически. Последовательности хроматических векторов, соответствующие
одинаковым структурным сегментам, усреднялись перед распознаванием аккордов. Эта
идея была продолжена в \cite{Cho2011}, где было предложено использовать метод
рекуррентного анализа для нахождения похожих друг на друга последовательностей
хроматических векторов и их взаимного сглаживания.

\subsection{Распознавание аккордов с использованием дополнительной информации}
\label{ssectT_recconstr}

\cite{Zhang2008}, \cite{Mcvicar2011}, \cite{Hrybyk2010}

\clearpage